% ----------------------------------------------------------------------------
%
%                           Probability and Statistics
%                                  Cookbook
%
% ----------------------------------------------------------------------------
%
% Copyright © Matthias Vallentin <matthias@berkeley.edu>, 2017
%

\documentclass[landscape]{article}

\usepackage{array}
\usepackage{amsmath,amssymb}
\usepackage{booktabs}
\usepackage{caption}
%\usepackage[nodayofweek]{datetime}
\usepackage{environ}
\usepackage{float}
\usepackage{enumitem}
\usepackage{fancyhdr}
\usepackage[landscape,margin=13mm,footskip=1pt,includefoot]{geometry}
\usepackage{graphicx}
\usepackage{hyperref}
\usepackage{multicol}
\usepackage{rotating}
\usepackage{tikz}
\usepackage{threeparttable}
\usepackage{url}
\usepackage{xspace}

% Document version, MAJOR.MINOR.PATCH. Please change with any modification
% according to semantic versioning practices:
%   - The major version changes when adding a new section or topic, or making a
%     substantial content change.
%   - The minor version changes for non-trivial fixes, corrections, or
%     improvements.
%   - The patch version changes for trivial fixes, such as typos in text or
%     formulas.
\newcommand{\version}{0.2.6}

% Probability and Statistics LaTeX shortcuts.
\input{probstat}

% TikZ tweaks
\usetikzlibrary{arrows,shapes}
\usetikzlibrary{decorations.pathreplacing}
\tikzstyle{every picture}+=[remember picture]
\tikzstyle{na} = [baseline=-.5ex]

% Move footnotes to the bottom-right corner
\pagestyle{fancy}
\fancyhf{} % clear all header and footer fields
\fancyhead{}
\fancyfoot[R]{\footnotesize \thepage}
\renewcommand{\headrulewidth}{0pt}

% Further document tweaks.
\parindent=0pt
\setitemize{itemsep=0.2mm,parsep=1pt}
\setenumerate{itemsep=0.2mm,parsep=1pt}

% A type of blue that doesn't look as aggressive as the default 'blue' but also
% distinguishes well from black while not appearing to light.
\definecolor{trueblue}{rgb}{0.0, 0.45, 0.81}

% Link style (hyperref package)
\hypersetup{
  colorlinks=true,        % false: boxed links; true: colored links
  linkcolor=black,        % color of internal links
  citecolor=trueblue,     % color of links to bibliography
  filecolor=trueblue,     % color of file links
  urlcolor=trueblue       % color of external links
}

% Personal
\def\email{info@statistics.zone}
\def\web{\url{http://statistics.zone/}}

% An itemize list with a title that avoids a break between title and list.
\newenvironment{titemize}[1]{
  \begin{minipage}[h]{\columnwidth}
    #1
    \begin{itemize}
}{
    \end{itemize}
  \end{minipage}
}

\begin{document}

\thispagestyle{empty}
\begin{center}
  \vspace*{\fill}
  \textsc{\Huge Probabilidad y Estad\'istica\\[2ex] \huge Fundamentos Generales }
  \vfill
  \footnotesize{
    Version \version\\[1ex]
    \today\\[1ex]
    \web\\[1ex]
    %Copyright \copyright{}
    \href{http://matthias.vallentin.net}{Matthias Vallentin}\\
  }
\end{center}

\newpage

\thispagestyle{empty}
\begin{multicols*}{3}
  \tableofcontents
  \vfill
  \hrule
  \vspace{5pt}
  {\footnotesize This cookbook integrates various topics in probability theory
  and statistics, based on literature~\cite{Hoel72,Wasserman03,Shumway06}
  and in-class material from courses of the statistics department at the
  University of California in Berkeley but also influenced by others
  \cite{Steger01,Steger02}. If you find errors or have suggestions for
  improvements, please get in touch at \web.}
\end{multicols*}

\newpage

\section{Distribution Overview}

\subsection{Discrete Distributions}

\begin{center}
\small
\begin{tabular}{@{}l*6{>{\begin{math}\displaystyle}c<{\end{math}}}@{}}
  \toprule &&&&&& \\[-2ex]
  & \text{Notation}\footnotemark
  & F_X(x) & f_X(x) & \E{X} & \V{X} & M_X(s) \\[1ex]

  \midrule

  Uniform & \unifd & \punifd & \dunifd &
  \frac{a+b}{2} & \frac{(b-a+1)^2-1}{12} &
  \frac{e^{as}-e^{-(b+1)s}}{s(b-a)} \\[3ex]

  Bernoulli & \bern & \pbern & \dbern &
  p & p(1-p) &
  1-p+pe^s \\[3ex]

  Binomial & \bin & I_{1-p}(n-x,x+1) & \dbin &
  np & np(1-p) &
  (1-p+pe^s)^n \\[3ex]

  Multinomial & \mult & & \dmult \quad \sum_{i=1}^k x_i = n&
  \left( {\begin{array}{*{20}{c}}
    {n{p_1}}\\
    \vdots \\
    {n{p_k}}
  \end{array}} \right) & \left( {\begin{array}{*{20}{c}}
    {n{p_1}(1 - {p_1})}&{ - n{p_1}{p_2}}\\
    { - n{p_2}{p_1}}& \ddots 
    \end{array}} \right) &
  \left( \sum_{i=0}^k p_i e^{s_i} \right)^n \\[3ex]

  Hypergeometric & \hyper &
  \approx \Phi\left(\displaystyle\frac{x-np}{\sqrt{np(1-p)}}\right) &
  \dhyper &
  \frac{nm}{N} & \frac{nm(N-n)(N-m)}{N^2(N-1)} & \\[3ex]

  Negative Binomial & \nbin & \pnbin & \dnbin &
  r\frac{1-p}{p} & r\frac{1-p}{p^2} &
  \left(\frac{p}{1-(1-p)e^s}\right)^r \\[3ex]

  Geometric & \geo &
  \pgeo \quad x\in\mathbb N^+ &
  \dgeo \quad x\in\mathbb N^+ &
  \frac{1}{p} & \frac{1-p}{p^2} &
  \frac{pe^s}{1-(1-p)e^s} \\[3ex]

  Poisson & \pois & \ppois & \dpois &
  \lambda & \lambda &
  e^{\lambda(e^s-1)}\\[3ex]

  \bottomrule
\end{tabular}
\end{center}

\footnotetext{We use the notation $\gamma(s,x)$ and $\Gamma(x)$ to refer to the
Gamma functions (see \S\ref{sec:math:gamma}), and use $\text{B}(x,y)$ and $I_x$
to refer to the Beta functions (see \S\ref{sec:math:beta}).}

\pagebreak

\begin{figure}[H]
  \includegraphics[scale=0.35]{figs/uniform-pmf.pdf}
  \includegraphics[scale=0.35]{figs/binomial-pmf.pdf}
  \includegraphics[scale=0.35]{figs/geometric-pmf.pdf}
  \includegraphics[scale=0.35]{figs/poisson-pmf.pdf}

  \includegraphics[scale=0.35]{figs/uniform-cdf-discrete.pdf}
  \includegraphics[scale=0.35]{figs/binomial-cdf.pdf}
  \includegraphics[scale=0.35]{figs/geometric-cdf.pdf}
  \includegraphics[scale=0.35]{figs/poisson-cdf.pdf}
\end{figure}

\subsection{Continuous Distributions}

\begin{threeparttable}
\small
%\newcolumntype{L}{>{\varwidth[c]{\linewidth}}l<{\endvarwidth}}
\newcolumntype{M}{>{\begin{math}\displaystyle}c<{\end{math}}}
\begin{tabular}{@{}l*6{M}@{}}
  \toprule &&&&&& \\[-2ex]
  & \text{Notation}
  & F_X(x) & f_X(x) & \E{X} & \V{X} & M_X(s) \\[1ex]

  \midrule

  Uniform & \unif & \punif & \dunif &
  \frac{a+b}{2} & \frac{(b-a)^2}{12} &
  \frac{e^{sb}-e^{sa}}{s(b-a)} \\[3ex]

  Normal & \norm &
  \Phi(x)=\displaystyle\int_{-\infty}^x \phi(t)\,dt &
  \phi(x)=\dnorm &
  \mu & \sigma^2 &
  \Exp{\mu s + \frac{\sigma^2s^2}{2}}\\[3ex]

  Log-Normal & \ln\norm&
  \frac{1}{2}+\frac{1}{2} \erf\left[\frac{\ln x-\mu}{\sqrt{2\sigma^2}}\right] &
  \frac{1}{x\sqrt{2\pi\sigma^2}} \Exp{-\frac{(\ln x - \mu)^2}{2\sigma^2}} &
  e^{\mu+\sigma^2/2} &
  (e^{\sigma^2}-1) e^{2\mu+\sigma^2} &
  \\[3ex]

  Multivariate Normal & \mvn & &
  (2\pi)^{-k/2} |\Sigma|^{-1/2} e^{-\frac{1}{2}(x-\mu)^T \Sigma^{-1}(x-\mu)} &
  \mu & \Sigma &
  \Exp{\mu^T s + \frac{1}{2} s^T \Sigma s}\\[3ex]

  Student's $t$ & \text{Student}(\nu)
  & I_x\left( \frac{\nu}{2},\frac{\nu}{2} \right)
  & \frac{\Gamma\left(\frac{\nu+1}{2}\right)}
    {\sqrt{\nu\pi}\Gamma\left(\frac{\nu}{2}\right)}
    \left(1+\frac{x^2}{\nu}\right)^{-(\nu+1)/2}
  & 0 \quad \nu  > 1
  & \begin{cases}
      \displaystyle\frac{\nu}{\nu-2} & \nu > 2 \\
      \infty & 1 < \nu \le 2
    \end{cases}
  & \\[3ex]

  Chi-square & \chisq &
  \frac{1}{\Gamma(k/2)} \gamma\left(\frac{k}{2}, \frac{x}{2}\right) &
  \frac{1}{2^{k/2} \Gamma(k/2)} x^{k/2-1} e^{-x/2}&
  k & 2k &
  (1-2s)^{-k/2} \; s<1/2\\[3ex]

  F & \text{F}(d_1,d_2) &
  I_\frac{d_1x}{d_1x+d_2}\left(\frac{d_1}{2},\frac{d_2}{2}\right) &
  \frac{\sqrt{\frac{(d_1x)^{d_1} d_2^{d_2}}{(d_1x+d_2)^{d_1+d_2}}}}
    {x\mathrm{B}\left(\frac{d_1}{2},\frac{d_1}{2}\right)} &
  \frac{d_2}{d_2-2} %\; d_2 > 2
  & \frac{2d_2^2(d_1+d_2-2)}{d_1(d_2-2)^2(d_2-4)} %\; d_2 > 4
  & \\[3ex]

  Exponential\tnote{$\ast$} & \ex & \pex & \dex &
  \beta & \beta^2 &
  \frac{1}{1-\frac{s}{\beta}} \left(s<\beta\right) \\[3ex]

  Gamma\tnote{$\ast$} & \gam &
  \frac{\gamma(\alpha,\beta x)}{\Gamma(\alpha)} & \dgamma &
  \frac{\alpha}{\beta} & \frac{\alpha}{\beta^2} &
  \left(\frac{1}{1-\frac{s}{\beta}} \right)^\alpha \left(s<\beta\right)\\[3ex]

  Inverse Gamma & \invgamma & \pinvgamma & \dinvgamma &
  \frac{\beta}{\alpha-1} \; \alpha>1 &
  \frac{\beta^2}{(\alpha-1)^2(\alpha-2)} \; \alpha > 2 &
  \frac{2(-\beta s)^{\alpha/2}}{\Gamma(\alpha)}K_\alpha
  \left( \sqrt{-4\beta s} \right)\\[3ex]

  Dirichlet & \dir & & \ddir &
  \frac{\alpha_i}{\sum_{i=1}^k \alpha_i} &
  \frac{\E{X_i}(1-\E{X_i})}{\sum_{i=1}^k\alpha_i + 1} & \\[3ex]

  Beta & \bet & I_x(\alpha,\beta)& \dbeta &
  \frac{\alpha}{\alpha+\beta} &
  \frac{\alpha\beta}{(\alpha+\beta)^2(\alpha+\beta+1)} &
  1+\sum_{k=1}^{\infty} \left( \prod_{r=0}^{k-1}
    \frac{\alpha+r}{\alpha+\beta+r} \right) \frac{s^k}{k!} \\[3ex]

  Weibull & \mathrm{Weibull}(\lambda, k) & 1 - e^{-(x/\lambda)^k} & \dweibull &
  \lambda \Gamma\left(1 + \frac{1}{k} \right) &
  \lambda^2 \Gamma\left(1 + \frac{2}{k}\right) - \mu^2 &
  \sum_{n=0}^\infty \frac{s^n \lambda^n}{n!} \Gamma\left(1+\frac{n}{k}\right)
  \\[3ex]

  Pareto & \mathrm{Pareto}(x_m, \alpha) &
  1 - \left(\frac{x_m}{x} \right)^\alpha \; x\ge x_m &
  \alpha\frac{x_m^\alpha}{x^{\alpha+1}} \quad x\ge x_m&
  \frac{\alpha x_m}{\alpha-1} \; \alpha>1 &
  \frac{x_m^2\alpha}{(\alpha-1)^2(\alpha-2)} \; \alpha>2 &
  \alpha(-x_m s)^\alpha \Gamma(-\alpha,-x_m s) \; s<0\\[3ex]

  \bottomrule
\end{tabular}
\begin{tablenotes}
\item[$\ast$] We use the \emph{rate} parameterization where
  $\beta=\frac{1}{\lambda}$. Some textbooks use $\beta$ as \emph{scale}
  parameter instead~\cite{Wasserman03}.
\end{tablenotes}
\end{threeparttable}

\begin{figure}[H]
  \includegraphics[scale=0.35]{figs/uniform-pdf.pdf}
  \includegraphics[scale=0.35]{figs/normal-pdf.pdf}
  \includegraphics[scale=0.35]{figs/lognormal-pdf.pdf}
  \includegraphics[scale=0.35]{figs/student-pdf.pdf}
  \includegraphics[scale=0.35]{figs/chisquare-pdf.pdf}
  \includegraphics[scale=0.35]{figs/f-pdf.pdf}
  \includegraphics[scale=0.35]{figs/exponential-pdf.pdf}
  \includegraphics[scale=0.35]{figs/gamma-pdf.pdf}
  \includegraphics[scale=0.35]{figs/invgamma-pdf.pdf}
  \includegraphics[scale=0.35]{figs/beta-pdf.pdf}
  \includegraphics[scale=0.35]{figs/weibull-pdf.pdf}
  \includegraphics[scale=0.35]{figs/pareto-pdf.pdf}
\end{figure}

\begin{figure}[H]
  \includegraphics[scale=0.35]{figs/uniform-cdf-continuous.pdf}
  \includegraphics[scale=0.35]{figs/normal-cdf.pdf}
  \includegraphics[scale=0.35]{figs/lognormal-cdf.pdf}
  \includegraphics[scale=0.35]{figs/student-cdf.pdf}
  \includegraphics[scale=0.35]{figs/chisquare-cdf.pdf}
  \includegraphics[scale=0.35]{figs/f-cdf.pdf}
  \includegraphics[scale=0.35]{figs/exponential-cdf.pdf}
  \includegraphics[scale=0.35]{figs/gamma-cdf.pdf}
  \includegraphics[scale=0.35]{figs/invgamma-cdf.pdf}
  \includegraphics[scale=0.35]{figs/beta-cdf.pdf}
  \includegraphics[scale=0.35]{figs/weibull-cdf.pdf}
  \includegraphics[scale=0.35]{figs/pareto-cdf.pdf}
\end{figure}

\begin{multicols*}{2}

\section{Probability Theory}

Definitions
\begin{itemize}
  \item Sample space $\Omega$
  \item Outcome (point or element) $\omega \in \Omega$
  \item Event $A \subseteq \Omega$
  \item $\sigma$-algebra $\mathcal{A}$
    \begin{enumerate}
      \item $\varnothing \in \mathcal{A}$
      \item $A_1,A_2,\dots, \in \mathcal{A}
        \imp \bigcup_{i=1}^\infty A_i \in \mathcal{A}$
      \item $A \in \mathcal{A} \imp \comp{A} \in \mathcal{A}$
    \end{enumerate}
  \item Probability Distribution $\prob$
    \begin{enumerate}
      \item $\Pr{A} \ge 0 \quad \forall A$
      \item $\Pr{\Omega} = 1$
      \item $\Pr{\displaystyle\bigsqcup_{i=1}^\infty A_i}
        = \displaystyle\sum_{i=1}^\infty \Pr{A_i}$
    \end{enumerate}
  \item Probability space $(\Omega,\mathcal{A},\prob)$
\end{itemize}

Properties
\begin{itemize}
  \item $\Pr{\varnothing} = 0$
  \item $B = \Omega \cap B = (A \cup \comp{A}) \cap B
    = (A \cap B) \cup (\comp{A} \cap B)$
  \item $\Pr{\comp{A}} = 1 - \Pr{A}$
  \item $\Pr{B} = \Pr{A \cap B} + \Pr{\comp{A} \cap B}$
  \item $\Pr{\Omega} = 1 \qquad \Pr{\varnothing} = 0$
  \item $\comp{\left(\bigcup_n A_n\right)} = \bigcap_n \comp{A_n}
    \quad
    \comp{\left(\bigcap_n A_n\right)} = \bigcup_n \comp{A_n}
    \qquad$
    \textsc{DeMorgan}
  \item $\Pr{\bigcup_n A_n}
    = 1 - \Pr{\bigcap_n \comp{A_n}}$
  \item $\Pr{A \cup B} = \Pr{A} + \Pr{B} - \Pr{A \cap B}\\[1ex]
    \imp \Pr{A \cup B} \le \Pr{A} + \Pr{B}$
  \item $\Pr{A \cup B}
    = \Pr{A \cap \comp{B}} + \Pr{\comp{A} \cap B} + \Pr{A \cap B}$
  \item $\Pr{A \cap \comp{B}} = \Pr{A} - \Pr{A \cap B}$
\end{itemize}

Continuity of Probabilities
\begin{itemize}
  \item $A_1 \subset A_2 \subset \dots \imp \limn \Pr{A_n} = \Pr{A}
    \quad\text{where } A = \bigcup_{i=1}^\infty A_i$
  \item $A_1 \supset A_2 \supset \dots \imp \limn \Pr{A_n} = \Pr{A}
    \quad\text{where } A = \bigcap_{i=1}^\infty A_i$
\end{itemize}

Independence \ind
\[A \ind B \eqv \Pr{A \cap B} = \Pr{A}\Pr{B}\]

Conditional Probability
\[\Pr{A \giv B} = \frac{\Pr{A \cap B}}{\Pr{B}} \qquad \Pr{B} > 0\]

Law of Total Probability
\[ \Pr{B} = \sum_{i=1}^n \Pr{B|A_i}\Pr{A_i}
  \qquad \Omega = \bigsqcup_{i=1}^n A_i\]

\textsc{Bayes' Theorem}
\[\Pr{A_i \giv B}
= \frac{\Pr{B \giv A_i}\Pr{A_i}}{\sum_{j=1}^n \Pr{B \giv A_j}\Pr{A_j}}
\qquad \Omega = \bigsqcup_{i=1}^n A_i\]

Inclusion-Exclusion Principle
\[\biggl|\bigcup_{i=1}^n A_i\biggr| = \sum_{r=1}^n(-1)^{r-1}
  \sum_{i \le i_1 < \dots < i_r \le n}\biggl|\bigcap_{j=1}^r A_{i_j}\biggr|\]

\section{Random Variables}

Random Variable (RV)
\[X: \Omega \to \R\]

Probability Mass Function (PMF)
\[f_X(x) = \Pr{X = x} = \Pr{\{\omega\in\Omega:X(\omega) = x\}}\]

Probability Density Function (PDF)
\[\Pr{a \le X \le b} = \int_a^b f(x)\dx\]

Cumulative Distribution Function (CDF)
\[F_X:\R \to [0,1] \qquad F_X(x) = \Pr{X \le x}\]

\begin{enumerate}
  \item Nondecreasing: $x_1 < x_2 \imp F(x_1) \le F(x_2)$
  \item Normalized: $\lim_{x\to -\infty} = 0$ and $\lim_{x\to \infty} = 1$
  \item Right-Continuous: $\lim_{y\downarrow x} F(y) = F(x)$
\end{enumerate}

\[\Pr{a\le Y\le b \giv X=x} = \int_a^b f_{Y|X}(y\giv x) dy \qquad a \le b\]
\[ f_{Y|X}(y\giv x) = \frac{f(x,y)}{f_X(x)} \]

Independence
\begin{enumerate}
  \item $\Pr{X \le x, Y \le y} = \Pr{X \le x}\Pr{Y \le y}$
  \item $f_{X,Y}(x,y) = f_X(x)f_Y(y)$
\end{enumerate}

\subsection{Transformations}

Transformation function
\[Z = \transform(X)\]

Discrete
\[f_Z(z) = \Pr{\transform(X) = z} = \Pr{\{x:\transform(x) = z\}}
= \Pr{X \in \transform^{-1}(z)} = \sum_{x \in \transform^{-1}(z)} \!\!\!f_X(x)\]

Continuous
\[F_Z(z) = \Pr{\transform(X) \le z} = \int_{A_z} f(x) \dx \quad
    \text{with } A_z = \{x:\transform(x) \le z\}\]

Special case if $\transform$ strictly monotone
\[f_Z(z)
    = f_X(\transform^{-1}(z))
      \left|\frac{d}{dz}\transform^{-1}(z)\right|
    = f_X(x)\left|\frac{dx}{dz}\right|
    = f_X(x)\frac{1}{|J|}\]

The Rule of the Lazy Statistician
\[\E{Z} = \int \transform(x) \dfx\]
\[\E{I_A(x)} = \int I_A(x) \dfx = \int_A \dfx = \Pr{X \in A}\]

Convolution
\begin{itemize}
  \item $ Z:=X+Y \qquad
    f_Z(z)=\displaystyle\int_{-\infty}^{\infty} f_{X,Y}(x,z-x)\,dx
    \;\stackrel{X,Y \ge 0}{=}\; \int_0^z f_{X,Y}(x,z-x)\,dx$
  \item $ Z:=|X-Y| \qquad
    f_Z(z)=\displaystyle2\int_0^\infty f_{X,Y}(x,z+x)\,dx$
    %\;\stackrel{X,Y \ge 0}{=}\; \int_0^\infty f_{X,Y}(x,z+x)\,dx$
  \item $ Z:=\displaystyle\frac{X}{Y} \qquad
    f_Z(z)=\displaystyle\int_{-\infty}^{\infty} |y| f_{X,Y}(yz,y)\,dy
    \;\stackrel{\ind}{=}\; \int_{-\infty}^{\infty} |y| f_X(yz)f_Y(y)\,dy$
\end{itemize}

%  \subsection{Joint Distribution}
%  \begin{itemize}
%    \item $f(x,y) = \Pr{X \le k, Y \le m)}
%      = \displaystyle\int_{-\infty}^k\int_{-\infty}^m f(x,y)\,dy\,dx$
%    \item $\Pr{a < X \le b, c < y \le d} = F(b,d) - F(a,d) - F(b,c) + F(a,c)$
%    \item $f_X(x) = \displaystyle\int_{-\infty}^\infty f(x,y)\,dy \qquad
%      f_Y(y) = \displaystyle\int_{-\infty}^\infty f(x,y)\,dx$
%  \end{itemize}

%  Order Statistics
%  \begin{itemize}
%    \item $U_i\ind U_j$ continuous \textsc{RVs} with common density $f$
%    \item $X_1(\omega) < \dots < X_n(\omega)$ permuted set of $U_i$'s
%    \item $X_k = $ \emph{k$^{th}$ order statistic}
%    \item $X_1(\omega) = \min(U_1(\omega),\dots,U_n(\omega))$
%    \item $X_n(\omega) = \max(U_1(\omega),\dots,U_n(\omega))$
%    \item $R(\omega) = X_n(\omega) - X_1(\omega)$
%  \end{itemize}

\section{Expectation}

Definition and properties
\begin{itemize}
  \item $\E{X} = \mu_X = \displaystyle \int x \dfx =
    \begin{cases}
      \displaystyle\sum_x xf_X(x) & \text{X discrete} \\\\
      \displaystyle\int xf_X(x)\dx & \text{X continuous}
    \end{cases}$
  \item $\Pr{X=c}=1 \imp \E{X} = c$
  \item $\E{cX} = c\,\E{X}$
  \item $\E{X+Y} = \E{X}+\E{Y}$
  \item $\E{XY} = \displaystyle\int_{X,Y} xy f_{X,Y}(x,y)\dfx\dfy$
  \item $\E{\transform(Y)} \neq \transform(\E{X}) \qquad$
    (cf.~\hyperref[jensen]{\textsc{Jensen} inequality})
  \item $\Pr{X \ge Y} = 1 \imp \E{X}\ge\E{Y}$
  \item $\Pr{X=Y} = 1 \imp \E{X}=\E{Y}$
%  \item $\Pr{\lvert Y\rvert\le c} = 1 \imp \E{Y}<\infty
%    \wedge \lvert\E{X}\rvert\le c$
  \item $\E{X} = \displaystyle\sum_{x=1}^\infty \Pr{X\ge x}$ \qquad X discrete
\end{itemize}

Sample mean
\[\samplemean = \frac{1}{n}\sum_{i=1}^n X_i\]

\begin{titemize}{Conditional expectation}
  \item $\E{Y\giv X=x} = \displaystyle\int y f(y\giv x)\dy$
  \item $\E{X} = \E{\E{X\giv Y}}$
  \item $\E[\transform(X,Y)\giv X=x]
    = \displaystyle\int_{-\infty}^\infty \transform(x,y)f_{Y|X}(y\giv x)\dx$
  \item $\E{\transform(Y,Z)\giv X=x} =
    \displaystyle\int_{-\infty}^\infty\transform(y,z)
    f_{(Y,Z)|X}(y,z\giv x)\,dy\,dz$
  \item $\E{Y+Z\giv X} = \E{Y\giv X} + \E{Z\giv X}$
  \item $\E{\transform(X)Y\giv X} = \transform(X)\E{Y\giv X}$
  \item $\E{Y\giv X} = c \imp \cov{X,Y}=0$
\end{titemize}

\section{Variance}

\begin{titemize}{Definition and properties}
  \item $\V{X} = \sigma_X^2 = \E{(X-\E{X})^2} = \E{X^2} - \E{X}^2$
  \item $\V{\displaystyle\sum_{i=1}^n X_i} =
    \displaystyle\sum_{i=1}^n \V{X_i} + \sum_{i\ne j}\cov{X_i,X_j}$
%    \stackrel{X_i \ind X_j}{=}\sum_{i=1}^n\V{X_i}$
  \item $\V{\displaystyle\sum_{i=1}^n X_i} =
    \displaystyle\sum_{i=1}^n\V{X_i} \quad$ if $X_i \ind X_j$
\end{titemize}

Standard deviation
\[\sd[X] = \sqrt{\V{X}} = \sigma_X\]

Covariance
\begin{itemize}
  \item $\cov{X,Y} = \E{(X-\E{X})(Y-\E{Y})} = \E{XY}-\E{X}\E{Y}$
  \item $\cov{X,a} = 0$
  \item $\cov{X,X} = \V{X}$
  \item $\cov{X,Y} = \cov{Y,X}$
  \item $\cov{aX,bY} = ab\cov{X,Y}$
  \item $\cov{X+a,Y+b} = \cov{X,Y}$
  \item $\cov{\displaystyle\sumin X_i, \sumjm Y_j}
    = \displaystyle\sumin\sumjm\cov{X_i, Y_j}$
\end{itemize}

Correlation
\[\corr{X,Y} = \displaystyle\frac{\cov{X,Y}}{\sqrt{\V{X}\V{Y}}}\]

Independence
\[X\ind Y \imp \corr{X,Y} = 0 \eqv \cov{X,Y} = 0 \eqv \E{XY}=\E{X}\E{Y}\]

Sample variance
\[\samplevar = \frac{1}{n-1}\sum_{i=1}^n(X_i-\samplemean)^2\]

Conditional variance
\begin{itemize}
  \item $\V{Y\giv X} = \E{(Y-\E{Y\giv X})^2\giv X} =\E{Y^2\giv X}-\E{Y\giv X}^2$
  \item $\V{Y} = \E{\V{Y\giv X}}+\V{\E{Y\giv X}}$
\end{itemize}

\section{Inequalities}

\textsc{Cauchy-Schwarz}
\[\E{XY}^2 \le \E{X^2}\E{Y^2}\]

\textsc{Markov}
\[\Pr{\transform(X) \ge t}\le\frac{\E{\transform(X)}}{t}\]

\textsc{Chebyshev}
\[\Pr{\lvert X-\E{X}\rvert \ge t} \le \frac{\V{X}}{t^2}\]

\textsc{Chernoff}
\[\Pr{X \ge (1+\delta)\mu}
\le \left(\frac{e^\delta}{(1+\delta)^{1+\delta}}\right) \quad \delta>-1\]

\textsc{Hoeffding}
\[X_1,\ldots,X_n \; \textrm{independent}
\;\wedge\; \Pr{X_i\in[a_i,b_i]} = 1 \;\wedge\; 1 \le i \le n \]
\[\Pr{\Xbar-\E{\Xbar} \ge t} \le e^{-2nt^2} \quad t>0 \]
\[\Pr{|\Xbar-\E{\Xbar}| \ge t} \le 2\Exp{-\frac{2n^2t^2}{\sumin(b_i-a_i)^2}}
\quad t>0\]

\textsc{Jensen}\label{jensen}
\[\E{\transform(X)} \ge \transform(\E{X}) \quad
  \transform \text{ convex}\]

\section{Distribution Relationships}

Binomial
\begin{itemize}
  \item $X_i \dist \bern \imp \displaystyle\sum_{i=1}^n X_i \dist \bin$
  \item $X\dist\bin, Y\dist\bin[m,p] \imp X+Y\dist\bin[n+m,p]$
  \item $\limn\bin = \pois[np] \qquad$ ($n$ large, $p$ small)
  \item $\limn\bin = \norm[np,np(1-p)] \qquad$
    ($n$ large, $p$ far from 0 and 1)
\end{itemize}

Negative Binomial
\begin{itemize}
  \item $ X\dist \nbin[1,p] = \geo $
  \item $ X\dist \nbin[r,p] = \sum_{i=1}^r \geo $
  \item $X_i\dist \nbin[r_i,p] \imp \sum X_i\dist \nbin[\sum r_i,p] $
  \item $X\dist \nbin[r,p].\; Y\dist \bin[s+r,p] \imp \Pr{X\le s} = \Pr{Y\ge r}$
\end{itemize}

Poisson
\begin{itemize}
  \item $X_i\dist\pois[\lambda_i] \wedge X_i \ind X_j
    \imp \displaystyle\sumin X_i \dist \pois[\displaystyle\sumin \lambda_i]$
  \item $X_i\dist\pois[\lambda_i] \wedge X_i \ind X_j
    \imp X_i\,\left|\displaystyle\sumjn X_j\right. \dist
   \bin[\displaystyle\sumjn X_j,\displaystyle\frac{\lambda_i}{\sumjn\lambda_j}]$
\end{itemize}

Exponential
\begin{itemize}
%    \item $\forall n \in \mathbb N^+: X_i\dist\ex{\lambda}
  \item $X_i\dist\ex \wedge  X_i \ind X_j
    \imp \displaystyle\sumin X_i\dist \gam[n,\beta]$
  \item Memoryless property: $\Pr{X>x+y\giv X>y}=\Pr{X>x}$
\end{itemize}

Normal
\begin{itemize}
  \item $X\dist \norm[\mu,\sigma^2]
    \imp \left(\frac{X-\mu}{\sigma}\right)\dist\norm[0,1] $
  \item $X\dist \norm[\mu,\sigma^2] \wedge Z = aX+b
    \imp Z\dist\norm[a\mu+b,a^2\sigma^2] $
  \item $X_i\dist\norm[\mu_i,\sigma_i^2] \wedge X_i \ind X_j
     \imp \sum_i X_i \dist \norm[\sum_i\mu_i,\sum_i\sigma_i^2]$
   \item $\Pr{a < X \le b}= \Phi\left(\frac{b-\mu}{\sigma}\right)
     - \Phi\left(\frac{a-\mu}{\sigma}\right) $
  \item $\Phi(-x) = 1 - \Phi(x) \qquad \phi'(x) = -x\phi(x) \qquad
    \phi''(x) = (x^2-1)\phi(x)$
  \item Upper quantile of $\norm[0,1]$: $z_{\alpha} = \Phi^{-1}(1-\alpha)$
\end{itemize}

Gamma
\begin{itemize}
  \item $X\dist\gam \eqv X/\beta \dist\gam[\alpha,1]$
  \item $\gam\dist \sum_{i=1}^\alpha\ex$
  \item $X_i\dist\gam[\alpha_i,\beta] \wedge X_i \ind X_j \imp
    \sum_i X_i\dist \gam[\sum_i \alpha_i,\beta]$
  \item $\displaystyle\frac{\Gamma(\alpha)}{\lambda^\alpha}
    = \displaystyle\int_0^\infty x^{\alpha-1} e^{-\lambda x} \dx$
\end{itemize}

Beta
\begin{itemize}
  \item $\displaystyle
    \frac{1}{\text{B}(\alpha,\beta)}x^{\alpha-1}(1-x)^{\beta-1}
    = \frac{\Gamma(\alpha+\beta)}{\Gamma(\alpha)\Gamma(\beta)}
    x^{\alpha-1}(1-x)^{\beta-1} $
  \item $\E{X^k}
    = \displaystyle\frac{\text{B}(\alpha+k,\beta)}{\text{B}(\alpha,\beta)}
    = \displaystyle\frac{\alpha+k-1}{\alpha+\beta+k-1}\E{X^{k-1}}$
  \item $\bet[1,1] \dist \unif[0,1]$
\end{itemize}

\section{Probability and Moment Generating Functions}

\begin{itemize}
  \item $G_X(t) = \E{t^X} \qquad |t| < 1$
  \item $M_X(t) = G_X(e^t) = \E{e^{Xt}}
    = \E{ \displaystyle\sum_{i=0}^\infty \frac{(Xt)^i}{i!}}
    = \displaystyle\sum_{i=0}^\infty \frac{\E{X^i}}{i!}\cdot t^i$
  \item $\Pr{X=0} = G_X(0)$
  \item $\Pr{X=1}=G_X'(0)$
  \item $\Pr{X=i} = \displaystyle\frac{G_X^{(i)}(0)}{i!}$
  \item $\E{X} = G_X'(1^-)$
  \item $\E{X^k} = M_X^{(k)}(0)$
  \item $\E{\displaystyle\frac{X!}{(X-k)!}} = G_X^{(k)}(1^-)$
  \item $\V{X} = G_X''(1^-) + G_X'(1^-)
    - \left(G_X'(1^-)\right)^2$
  \item $G_X(t) = G_Y(t) \imp X \stackrel{d}{=} Y$
\end{itemize}

\section{Multivariate Distributions}

\subsection{Standard Bivariate Normal}

Let $X,Y\dist\norm[0,1] \wedge X\ind Z$ where
$Y = \rho X + \sqrt{1-\rho^2}Z$\\

Joint density
\[
f(x,y) = \frac{1}{2 \pi \sqrt{1-\rho^2}}
\Exp{-\frac{x^2 + y^2 - 2\rho x y}{2 (1-\rho^2)}}
\]

Conditionals
\[
(Y\giv X=x) \dist \norm[\rho x,1-\rho^2] \qquad\text{and}\qquad
(X\giv Y=y) \dist \norm[\rho y,1-\rho^2]
\]

Independence
\[X \ind Y \eqv \rho = 0\]

\subsection{Bivariate Normal}
% - http://www.athenasc.com/Bivariate-Normal.pdf
% - http://mathworld.wolfram.com/BivariateNormalDistribution.html

Let $X\dist\norm[\mu_x,\sigma_x^2]$
  and $Y\dist\norm[\mu_y,\sigma_y^2]$.
\[f(x,y) = \frac{1}{2 \pi \sigma_x \sigma_y \sqrt{1-\rho^2}}
\Exp{-\frac{z}{2 (1-\rho^2)}}\]
\[ z =
  \left[
  \left(\frac{x-\mu_x}{\sigma_x}\right)^2
    + \left(\frac{y-\mu_y}{\sigma_y}\right)^2
    - 2\rho\left(\frac{x-\mu_x}{\sigma_x}\right)
      \left(\frac{y-\mu_y}{\sigma_y}\right)
  \right]
\]

Conditional mean and variance
\[\E{X\giv Y} = \E{X} + \rho\frac{\sigma_X}{\sigma_Y}(Y-\E{Y})\]
\[\V{X\giv Y} = \sigma_X \sqrt{1-\rho^2}\]

\subsection{Multivariate Normal}

Covariance matrix $\Sigma$ \quad (Precision matrix $\Sigma^{-1}$)
\[\Sigma =
  \begin{pmatrix}
  \V{X_1} & \cdots & \cov{X_1,X_k} \\
  \vdots & \ddots & \vdots \\
  \cov{X_k,X_1} & \cdots & \V{X_k}
  \end{pmatrix}\]

If $X \dist \norm[\mu,\Sigma]$,
\[f_X(x) = (2\pi)^{-n/2} \left|\Sigma\right|^{-1/2}
\Exp{-\frac{1}{2}(x-\mu)^T\Sigma^{-1}(x-\mu)} \]

Properties
\begin{itemize}
  \item $Z \dist \norm[0,1] \wedge X = \mu+\Sigma^{1/2}Z
    \imp X \dist \norm[\mu,\Sigma]$
  \item $X \dist \norm[\mu,\Sigma] \imp \Sigma^{-1/2}(X-\mu) \dist \norm[0,1]$
  \item $X \dist \norm[\mu,\Sigma] \imp AX \dist \norm[A\mu, A\Sigma A^T]$
  \item $X \dist \norm[\mu,\Sigma] \wedge \|a\| = k
    \imp a^TX \dist \norm[a^T\mu, a^T\Sigma a]$
\end{itemize}

\section{Convergence}

Let $\{X_1,X_2,\ldots\}$ be a sequence of \rv's and let $X$ be another \rv.
Let $F_n$ denote the \cdf of $X_n$ and let $F$ denote the \cdf of $X$.

Types of Convergence
\begin{enumerate}
  \item In distribution (weakly, in law): $X_n \dconv X$
    \[\limn F_n(t) = F(t) \qquad
      \forall t \text{ where } F \text{ continuous}\]
  \item In probability: $X_n \pconv X$
    \[(\forall \varepsilon > 0) \;
    \lim_{n\to\infty} \Pr{|X_n -X| > \varepsilon} = 0\]
  \item Almost surely (strongly): $X_n \asconv X$
    \[\Pr{\limn X_n=X} = \Pr{\omega\in\Omega: \limn X_n(\omega)=X(\omega)}=1\]
  \item In quadratic mean ($L_2$): $X_n \qmconv X$
    \[\lim_{n\to\infty} \E{(X_n - X)^2} = 0\]
\end{enumerate}

Relationships
\begin{itemize}
  \item $X_n \qmconv X \imp X_n \pconv X \imp X_n \dconv X$
  \item $X_n \asconv X \imp X_n \pconv X$
  \item $X_n \dconv X \wedge (\exists c \in \R) \; \Pr{X=c} = 1
    \imp X_n \pconv X$
  \item $X_n \pconv X \wedge Y_n \pconv Y
    \imp X_n + Y_n \pconv X + Y$
  \item $X_n \qmconv X \wedge Y_n \qmconv Y
    \imp X_n + Y_n \qmconv X + Y$
  \item $X_n \pconv X \wedge Y_n \pconv Y
    \imp X_nY_n \pconv XY$
  \item $X_n \pconv X \imp \transform(X_n) \pconv \transform(X)$
  \item $X_n \dconv X \imp \transform(X_n) \dconv \transform(X)$
  \item $X_n \qmconv b \eqv \lim_{n\to\infty} \E{X_n}=b
    \wedge \lim_{n\to\infty} \V{X_n} = 0$
  \item $X_1,\dots,X_n\; \iid \wedge \E{X}=\mu \wedge \V{X}<\infty
    \eqv \samplemean \qmconv \mu$
\end{itemize}

\textsc{Slutzky's Theorem}
\begin{itemize}
  \item $X_n \dconv X \text{ and } Y_n \pconv c
    \imp X_n + Y_n \dconv X + c$
  \item $X_n \dconv X \text{ and } Y_n \pconv c
    \imp X_nY_n \dconv cX$
  \item In general: $X_n \dconv X \text{ and } Y_n \dconv Y
    \nimp X_n + Y_n \dconv X + Y$
\end{itemize}

\subsection{Law of Large Numbers (LLN)}

Let $\{X_1,\ldots,X_n\}$ be a sequence of \iid \rv's, $\E{X_1}=\mu$.

Weak (WLLN)
\[\samplemean \pconv \mu \qquad n\to\infty\]

Strong (SLLN)
\[\samplemean \asconv \mu \qquad n\to\infty\]

\subsection{Central Limit Theorem (CLT)}

Let $\{X_1,\ldots,X_n\}$ be a sequence of \iid \rv's, $\E{X_1}=\mu$, and
$\V{X_1} = \sigma^2$.\\

\[ Z_n
  := \displaystyle\frac{\samplemean-\mu}{\sqrt{\V{\samplemean}}}
  = \displaystyle\frac{\sqrt{n}(\samplemean - \mu)}{\sigma}
  \dconv Z \qquad \text{where } Z\dist \norm[0,1]\]
\[\lim_{n\to\infty} \Pr{Z_n \le z} = \Phi(z) \qquad z \in \mathbb R\]

CLT notations
\begin{align*}
Z_n &\approx \norm[0,1] \\
\samplemean &\approx \norm[\mu,\frac{\sigma^2}{n}] \\
\samplemean - \mu &\approx \norm[0,\frac{\sigma^2}{n}] \\
\sqrt{n}(\samplemean - \mu) &\approx \norm[0,\sigma^2] \\
\frac{\sqrt{n}(\samplemean - \mu)}{\sigma} &\approx \norm[0,1] \\
\end{align*}

Continuity correction
\[\Pr{\samplemean \le x} \approx
  \Phi\left(\displaystyle\frac{x+\frac{1}{2}-\mu}{\sigma/\sqrt{n}}\right)\]
\[\Pr{\samplemean \ge x} \approx
  1-\Phi\left(\displaystyle\frac{x-\frac{1}{2}-\mu}{\sigma/\sqrt{n}}\right)\]

Delta method
\[Y_n \approx \norm[\mu,\frac{\sigma^2}{n}] \imp
\transform(Y_n) \approx
\norm[\transform(\mu),
  \left(\transform'(\mu)\right)^2\frac{\sigma^2}{n}]\]

\section{Statistical Inference}

Let $X_1,\cdots,X_n \distiid F$ if not otherwise noted.

\subsection{Point Estimation}

\begin{itemize}
  \item Point estimator $\that_n$ of $\theta$ is a \rv:
    $\that_n = g(X_1,\dots,X_n)$
  \item $\bias(\that_n) = \E{\that_n}-\theta$
  \item Consistency: $\that_n \pconv \theta$
  \item Sampling distribution: $F(\that_n)$
  \item Standard error: $\se(\that_n) = \sqrt{\V{\that_n}}$
  \item Mean squared error: $\mse = \E{(\that_n-\theta)^2}
    = \bias(\that_n)^2 + \V{\that_n}$
  \item $\limn \bias(\that_n) = 0 \wedge \limn\se(\that_n) = 0
    \imp \that_n$ is consistent
  \item Asymptotic normality:
    $\displaystyle\frac{\that_n-\theta}{\se} \dconv \norm[0,1]$
  \item \textsc{Slutzky's Theorem} often lets us replace $\se(\that_n)$ by some
    (weakly) consistent estimator $\shat_n$.
\end{itemize}

\subsection{Normal-Based Confidence Interval}

Suppose $\that_n \approx \norm[\theta,\sehat^2]$.
Let $\zat = \Phi^{-1}(1-(\alpha/2))$,
i.e., $\Pr{Z > \zat} = \alpha/2$ and $\Pr{-\zat < Z < \zat} = 1-\alpha$
where $Z\dist\norm[0,1]$.
Then \[C_n = \that_n \pm \zat\sehat\]

\subsection{Empirical distribution}

Empirical Distribution Function (ECDF)
\[\Fnhat(x) = \displaystyle\frac{\sumin I(X_i \le x)}{n}\]
\[I(X_i \le x) = \begin{cases}
  1 & X_i \le x \\
  0 & X_i > x
\end{cases}\]

Properties (for any fixed $x$)
\begin{itemize}
  \item $\E{\Fnhat} = F(x)$
  \item $\V{\Fnhat} = \displaystyle\frac{F(x)(1-F(x))}{n}$
  \item $\mse = \displaystyle\frac{F(x)(1-F(x))}{n} \dconv 0$
  \item $\Fnhat \pconv F(x)$
\end{itemize}

\textsc{Dvoretzky-Kiefer-Wolfowitz} (DKW) inequality ($X_1,\dots,X_n\dist F$)
\[\Pr{\sup_x\left|F(x)-\Fnhat(x)\right| > \varepsilon} =
  2e^{-2n\varepsilon^2}\]

Nonparametric $1-\alpha$ confidence band for $F$
\begin{align*}
  L(x) &= \max\{\Fnhat-\epsilon_n, 0\} \\
  U(x) &= \min\{\Fnhat+\epsilon_n, 1\} \\
  \epsilon &=
    \sqrt{\displaystyle\frac{1}{2n}\log\left( \frac{2}{\alpha} \right)} \\
\end{align*}
\[\Pr{L(x) \le F(x) \le U(x) \;\forall x} \ge 1-\alpha\]

\subsection{Statistical Functionals}

\begin{itemize}
  \item Statistical functional: $T(F)$
  \item Plug-in estimator of $\theta = (F)$: $\that_n = T(\Fnhat)$
  \item Linear functional: $T(F) = \int \transform(x)\dfx$
  \item Plug-in estimator for linear functional: \\
    \[T(\Fnhat)
      = \displaystyle\int \transform(x)\dfhatx
      = \frac{1}{n}\sumin \transform(X_i)\]
  \item Often: $T(\Fnhat) \approx \norm[T(F),\sehat^2]$ \imp
    $T(\Fnhat) \pm \zat\sehat$
  \item $p^\mathrm{th}$ quantile: $F^{-1}(p) = \inf\{x:F(x) \ge p\}$
  \item $\mhat = \samplemean$
  \item $\shat^2 = \displaystyle\frac{1}{n-1}\sumin
    (X_i-\samplemean)^2$
  \item  $\khat =
    \displaystyle\frac{\frac{1}{n}\sumin(X_i-\mhat)^3}{\shat^3}$
  \item $\rhohat = \displaystyle\frac{\sumin(X_i-\samplemean)(Y_i-\bar{Y}_n)}%
    {\sqrt{\sumin(X_i-\samplemean)^2}\sqrt{\sumin(Y_i-\bar{Y}_n)^2}}$
\end{itemize}

\section{Parametric Inference}

Let $\mathfrak{F} = \bigl\{ f(x;\theta) : \theta\in\Theta \bigr\}$ be a
parametric model with parameter space $\Theta \subset \R^k$ and parameter
$\theta = (\theta_1,\dots,\theta_k)$.

\subsection{Method of Moments}

$j^{\mathrm{th}}$ moment
\[\alpha_j(\theta) = \E{X^j} = \displaystyle\int x^j \dfx\]

$j^{\mathrm{th}}$ sample moment
\[\ahat_j = \displaystyle\frac{1}{n}\sumin X_i^j\]

Method of Moments estimator (MoM)
\begin{align*}
  \alpha_1(\theta) &= \ahat_1 \\
  \alpha_2(\theta) &= \ahat_2 \\
  \vdots &= \vdots \\
  \alpha_k(\theta) &= \ahat_k
\end{align*}

\begin{titemize}{Properties of the MoM estimator}
  \item $\that_n$ exists with probability tending to 1
  \item Consistency: $\that_n \pconv \theta$
  \item Asymptotic normality:
    \[\sqrt{n}(\that-\theta) \dconv \norm[0,\Sigma]\]
    where $\Sigma = g\E{YY^T}g^T$, $Y = (X,X^2,\dots,X^k)^T$,\\
    $g = (g_1,\dots,g_k)$ and
    $g_j = \frac{\partial}{\partial\theta}\alpha_j^{-1}(\theta)$
\end{titemize}

\subsection{Maximum Likelihood}

Likelihood: $\Lln : \Theta \to [0,\infty)$
  \[\Lln(\theta) = \displaystyle\prodin f(X_i;\theta)\] \\
Log-likelihood
  \[\lln(\theta) = \log \Lln(\theta) = \sumin \log f(X_i;\theta)\]
Maximum likelihood estimator (\mle)
\[\Lln(\that_n) = \sup_\theta \Lln(\theta)\]

Score function
\[s(X;\theta) = \frac{\partial}{\partial\theta}\log f(X;\theta)\]

Fisher information
\[I(\theta) = \V[\theta]{s(X;\theta)}\]
\[I_n(\theta) = nI(\theta)\]
Fisher information (exponential family)
\[I(\theta) = \E[\theta]{-\frac{\partial}{\partial\theta} s(X;\theta)}\]
Observed Fisher information
\[I_n^{obs}(\theta)
  = -\frac{\partial^2}{\partial\theta^2} \sumin\log f(X_i;\theta)\]

Properties of the \mle
\begin{itemize}
  \item Consistency: $\that_n \pconv \theta$
  \item Equivariance:
    $\that_n$ is the \mle
    \imp $\transform(\that_n)$ is the \mle of $\transform(\theta)$
  \item Asymptotic optimality (or efficiency), i.e., smallest variance for
    large samples. If $\ttil_n$ is any other estimator, the asymptotic relative
    efficiency is:
    \begin{enumerate}
      \item $\se \approx \sqrt{1/I_n(\theta)}$
        \[\frac{(\that_n - \theta)}{\se} \dconv \norm[0,1]\]
      \item $\sehat \approx \sqrt{1/I_n(\that_n)}$
        \[\frac{(\that_n - \theta)}{\sehat} \dconv \norm[0,1]\]
    \end{enumerate}
  \item Asymptotic optimality
    \[\are(\ttil_n, \that_n)
      = \frac{\V{\that_n}}{\V{\ttil_n}}
      \le 1\]
  \item Approximately the Bayes estimator
\end{itemize}

\subsubsection{Delta Method}
If $\tau=\transform(\that)$
where \transform is differentiable and $\transform'(\theta) \neq 0$:
\[\frac{(\widehat{\tau}_n-\tau)}{\sehat(\widehat{\tau})} \dconv \norm[0,1]\]
where $\widehat{\tau} = \transform(\that)$
is the \mle of $\tau$ and
\[\sehat = \left|\transform'(\that)\right|\sehat(\that_n)\]

\subsection{Multiparameter Models}

Let $\theta=(\theta_1,\dots,\theta_k)$
and $\that= (\that_1,\dots,\that_k)$ be the \mle.

\[H_{jj} = \frac{\partial^2 \lln}{\partial\theta^2} \qquad
H_{jk} = \frac{\partial^2 \lln}{\partial\theta_j\partial\theta_k} \qquad\]

Fisher information matrix
\[I_n(\theta) = -\begin{bmatrix}
  \E[\theta]{H_{11}} & \cdots & \E[\theta]{H_{1k}} \\
  \vdots & \ddots & \vdots \\
  \E[\theta]{H_{k1}} & \cdots & \E[\theta]{H_{kk}}
\end{bmatrix}\]

Under appropriate regularity conditions
\[(\that-\theta) \approx \norm[0,J_n]\]
with $J_n(\theta) = I_n^{-1}$.
Further, if $\that_j$ is the $j^{\mathrm{th}}$ component of $\theta$, then
\[\frac{(\that_j-\theta_j)}{\sehat_j} \dconv \norm[0,1]\] where $\sehat_j^2 =
J_n(j,j)$ and $\cov{\that_j,\that_k} = J_n(j,k)$

\subsubsection{Multiparameter delta method}

Let $\tau = \transform(\theta_1,\dots,\theta_k)$
and let the gradient of \transform be
\[\nabla\transform = \begin{pmatrix}
  \displaystyle\frac{\partial\transform}{\partial\theta_1} \\
  \vdots \\
  \displaystyle\frac{\partial\transform}{\partial\theta_k} \\
\end{pmatrix}\]
Suppose $\bigl.\nabla\transform\bigr|_{\theta=\that} \neq 0$ and
$\widehat{\tau} = \transform(\that)$. Then,
\[\frac{(\widehat{\tau}-\tau)}{\sehat(\widehat{\tau})} \dconv \norm[0,1]\]
where
\[\sehat(\widehat{\tau}) = \sqrt{\left( \widehat{\nabla}\transform \right)^T
\widehat{J}_n\left( \widehat{\nabla}\transform \right)}\]
and $\widehat{J}_n = J_n(\that)$ and $\widehat{\nabla}\transform =
\bigl.\nabla\transform\bigr|_{\theta=\that}$.

\subsection{Parametric Bootstrap}

Sample from $f(x;\that_n)$ instead of from $\Fnhat$, where $\that_n$ could be
the \mle or method of moments estimator.

\section{Hypothesis Testing}

\[H_0:\theta\in\Theta_0 \qquad\text{versus}\qquad H_1:\theta\in\Theta_1\]

Definitions
\begin{itemize}
  \item Null hypothesis $H_0$
  \item Alternative hypothesis $H_1$
  \item Simple hypothesis $\theta = \theta_0$
  \item Composite hypothesis $\theta > \theta_0$ or $\theta < \theta_0$
  \item Two-sided test:
    $H_0:\theta=\theta_0 \quad\text{versus}\quad H_1:\theta\neq\theta_0$
  \item One-sided test:
    $H_0:\theta\le\theta_0 \quad\text{versus}\quad H_1:\theta>\theta_0$
%  \[H_0:\theta\ge\theta_0 \qquad\text{versus}\qquad H_1:\theta<\theta_0\]
  \item Critical value $c$
  \item Test statistic $T$
  \item Rejection region $R = \left\{ x: T(x) > c \right\}$
  \item Power function $\beta(\theta) = \Pr{X \in R}$
  \item Power of a test: $1 - \Pr{\text{Type II error}} = 1-\beta
    = \displaystyle\inf_{\theta \in \Theta_1} \beta(\theta)$
  \item Test size: $\alpha = \Pr{\text{Type I error}}
    = \displaystyle\sup_{\theta\in\Theta_0}\beta(\theta)$
\end{itemize}

\centering
\begin{tabular}{l|cc}
  & \textsf{Retain} $H_0$ & \textsf{Reject} $H_0$ \\
  \hline
  $H_0$ \textsf{true} & $\surd$ & Type I Error ($\alpha$)\\
  $H_1$ \textsf{true} & Type II Error ($\beta$) &
  $\surd$ (power) \\
\end{tabular}

\raggedright
p-value
\begin{itemize}
  \item p-value $= \sup_{\theta\in\Theta_0} \Pr[\theta]{T(X) \ge T(x)}
                      = \inf\bigl\{ \alpha: T(x) \in R_\alpha \bigr\}$
  \item p-value $= \sup_{\theta\in\Theta_0}
    \underbrace{\Pr[\theta]{T(X^\star) \ge T(X)}}_{1-F_\theta(T(X))
          \quad \text{since } T(X^\star) \dist F_\theta}
                      = \inf\bigl\{ \alpha: T(X) \in R_\alpha \bigr\}$
\end{itemize}

\centering
\begin{tabular}{ll}
  \textsf{p-value} & \textsf{evidence} \\
  \hline
  $< 0.01$      & very strong evidence against $H_0$ \\
  $0.01 - 0.05$ & strong evidence against $H_0$ \\
  $0.05 - 0.1$  & weak evidence against $H_0$ \\
  $> 0.1$       & little or no evidence against $H_0$ \\
\end{tabular}

\raggedright
Wald test
\begin{itemize}
  \item Two-sided test
  \item Reject $H_0$ when $|W| > \zat$ where
    $W = \displaystyle\frac{\that - \theta_0}{\sehat}$
  \item $\Pr{|W| > \zat} \conv \alpha$
  \item p-value $= \Pr[\theta_0]{|W| > |w|}
                 \approx \Pr{|Z| > |w|}
                 = 2\Phi(-|w|)$
\end{itemize}

Likelihood ratio test
\begin{itemize}
 \item $T(X) = \displaystyle\frac{\sup_{\theta\in\Theta}\Lln(\theta)}%
                                 {\sup_{\theta\in\Theta_0}\Lln(\theta)}
             = \frac{\Lln(\that_n)}{\Lln(\that_{n,0})}$
 \item $\lambda(X) = 2\log T(X) \dconv \chi_{r-q}^2$
   where $\displaystyle\sum_{i=1}^k Z_i^2 \dist \chi_k^2$ and
    $Z_1,\dots,Z_k \distiid \norm[0,1]$
  \item p-value $= \Pr[\theta_0]{\lambda(X) > \lambda(x)}
                 \approx \Pr{\chi_{r-q}^2 > \lambda(x)}$
\end{itemize}

\begin{titemize}{Multinomial LRT}
  \item \mle:
    $\phat_n = \displaystyle \left(\frac{X_1}{n},\dots,\frac{X_k}{n}\right)$
 \item $T(X) = \displaystyle \frac{\Lln(\phat_n)}{\Lln(p_0)}
   = \prod_{j=1}^k \left( \frac{\phat_j}{p_{0j}} \right)^{X_j}$
 \item $\lambda(X) = \displaystyle 2\sum_{j=1}^k X_j \log
    \left( \frac{\phat_j}{p_{0j}} \right) \dconv \chi_{k-1}^2$
 \item The approximate size $\alpha$ LRT rejects $H_0$ when
  $\lambda(X) \ge \chi_{k-1,\alpha}^2$
\end{titemize}

Pearson Chi-square Test
\begin{itemize}
  \item $T = \displaystyle \sum_{j=1}^k \frac{(X_j-\E{X_j})^2}{\E{X_j}}$
    where $\E{X_j} = np_{0j}$ under $H_0$
  \item $T \dconv \chi_{k-1}^2$
  \item p-value $= \Pr{\chi_{k-1}^2 > T(x)}$
  \item Faster $\dconv X_{k-1}^2$ than LRT, hence preferable for small $n$
\end{itemize}

Independence testing
\begin{itemize}
  \item $I$ rows, $J$ columns,
    $\mathbf{X}$ multinomial sample of size $n=I*J$
  \item {\mle}s unconstrained: $\phat_{ij} = \frac{X_{ij}}{n}$
  \item {\mle}s under $H_0$:
    $\phat_{0ij} = \phat_{i\cdot}\phat_{\cdot j}
    = \frac{X_{i\cdot}}{n} \frac{X_{\cdot j}}{n}$
  \item LRT: $\lambda = 2\sum_{i=1}^I\sum_{j=1}^J X_{ij}
    \log\left( \frac{nX_{ij}}{X_{i\cdot}X_{\cdot j}}\right)$
  \item PearsonChiSq: $T = \sum_{i=1}^I\sum_{j=1}^J
    \frac{(X_{ij}-\E{X_{ij}})^2}{\E{X_{ij}}}$
  \item LRT and Pearson $\dconv \chisq{\nu}$,
    where $\nu=(I-1)(J-1)$
\end{itemize}

\section{Exponential Family}

Scalar parameter
\begin{align*}
f_X(x \giv \theta)
&= h(x) \Exp{\eta(\theta)T(x) - A(\theta)} \\
&= h(x) g(\theta) \Exp{\eta(\theta)T(x)}
\end{align*}

Vector parameter
\begin{align*}
f_X(x \giv \theta)
&= h(x) \Exp{\sum_{i=1}^s\eta_i(\theta)T_i(x) - A(\theta)} \\
&= h(x) \Exp{\eta(\theta)\cdot T(x) - A(\theta)} \\
&= h(x) g(\theta)\Exp{\eta(\theta)\cdot T(x)}
\end{align*}

Natural form
\begin{align*}
f_X(x \giv \eta)
&= h(x) \Exp{\eta\cdot \mathbf{T}(x) - A(\eta)} \\
&= h(x) g(\eta) \Exp{\eta\cdot \mathbf{T}(x)} \\
&= h(x) g(\eta) \Exp{\eta^T \mathbf{T}(x)}
\end{align*}

\section{Bayesian Inference}

\textsc{Bayes' Theorem}
\[f(\theta\giv x)
= \frac{f(x\giv\theta)f(\theta)}{f(x^n)}
= \frac{f(x\giv\theta)f(\theta)}
       {\int f(x\giv\theta)f(\theta)\,d\theta} \propto \Lln(\theta)f(\theta)\]

Definitions
\begin{itemize}
  \item $X^n = (X_1,\ldots,X_n)$
  \item $x^n = (x_1,\ldots,x_n)$
  \item Prior density $f(\theta)$
  \item Likelihood $f(x^n \giv \theta)$: joint density of the data\\
    In particular, $X^n$ \iid \imp $f(x^n\giv\theta) =
    \displaystyle\prodin f(x_i\giv\theta) = \Lln(\theta)$
  \item Posterior density $f(\theta\giv x^n)$
  \item Normalizing constant $c_n = f(x^n)
    = \int f(x\giv\theta)f(\theta)\,d\theta$
  \item Kernel: part of a density that depends on $\theta$
  \item Posterior mean $\bar\theta_n
    = \int\theta f(\theta\giv x^n)\,d\theta
    = \frac{\int\theta\Lln(\theta)f(\theta)d\theta}
           {\int\Lln(\theta)f(\theta)\,d\theta}$
\end{itemize}

\subsection{Credible Intervals}

Posterior interval
\[\Pr{\theta\in (a,b)\giv x^n}
= \int_{a}^{b} f(\theta\giv x^n)\,d\theta
= 1-\alpha\]

Equal-tail credible interval
\[\int_{-\infty}^{a} f(\theta\giv x^n)\,d\theta
= \int_{b}^{\infty} f(\theta\giv x^n)\,d\theta = \alpha/2\]

Highest posterior density (HPD) region $R_n$
\begin{enumerate}
  \item $\Pr{\theta\in R_n} = 1-\alpha$
  \item $R_n = \left\{ \theta: f(\theta\giv x^n) > k \right\}$ for some $k$
\end{enumerate}
$R_n$ is unimodal \imp $R_n$ is an interval

\subsection{Function of parameters}
Let $\tau = \transform(\theta)$ and $A =
\left\{ \theta:\transform(\theta) \le \tau \right\}$.

Posterior CDF for $\tau$
\[H(r\giv x^n)
= \Pr{\transform(\theta) \le \tau\giv x^n}
= \int_A f(\theta \giv x^n)\,d\theta\]
Posterior density
\[h(\tau\giv x^n) = H'(\tau\giv x^n)\]
Bayesian delta method
\[\tau\giv X^n \approx \norm[\transform(\that),
\sehat\left|\transform'(\that)\right|]\]

\subsection{Priors}

Choice
\begin{itemize}
  \item Subjective Bayesianism: prior should incorporate as much detail as
    possible the research's a priori knowledge---via \emph{prior elicitation}
  \item Objective Bayesianism: prior should incorporate as little detail as
    possible (\emph{non-informative} prior)
  \item Robust Bayesianism: consider various priors and determine
    \emph{sensitivity} of our inferences to changes in the prior
\end{itemize}

Types
\begin{itemize}
  \item Flat: $f(\theta) \propto constant$
  \item Proper: $\int_{-\infty}^{\infty} f(\theta)\,d\theta = 1$
  \item Improper: $\int_{-\infty}^{\infty} f(\theta)\,d\theta = \infty$
  \item \textsc{Jeffrey}'s Prior (transformation-invariant):
    \[f(\theta) \propto \sqrt{I(\theta)} \qquad
    f(\theta) \propto \sqrt{\det(I(\theta))}\]
  \item Conjugate: $f(\theta)$ and $f(\theta\giv x^n)$
    belong to the same parametric family
\end{itemize}

\subsubsection{Conjugate Priors}

\begin{tabular}{|l|p{.23\columnwidth}|p{.45\columnwidth}|}
  \hline
  \multicolumn{3}{|c|}{Continuous likelihood (subscript $c$ denotes constant)}\\
  \hline && \\[-2ex]
  Likelihood & Conjugate prior &
    \text{Posterior hyperparameters} \\[1ex]

  \hline && \\[-2ex]

  $\unif[0,\theta]$ & Pareto($x_m,k$) &
  $\displaystyle\max\left\{ x_{(n)}, x_m \right\}, k+n$ \\

  $\ex[\lambda]$ & $\gam[\alpha,\beta]$ &
  $\alpha + n, \beta + \displaystyle\sumin x_i$\\[3ex]

  $\norm[\mu,\sigma_c^2]$ & $\norm[\mu_0,\sigma_0^2]$ &
  $\displaystyle
  \left(\frac{\mu_0}{\sigma_0^2} + \frac{\sumin x_i}{\sigma_c^2} \right) /
  \left( \frac{1}{\sigma_0^2} + \frac{n}{\sigma_c^2} \right)$,
  $\displaystyle\left(\frac{1}{\sigma_0^2} + \frac{n}{\sigma_c^2}\right)^{-1}$
  \\[2ex]

  $\norm[\mu_c,\sigma^2]$ & Scaled Inverse Chi-square($\nu,\sigma_0^2$) &
  $\nu + n$, $\displaystyle \frac{\nu\sigma_0^2 + \sumin(x_i-\mu)^2}{\nu + n}$
  \\[4ex]

  $\norm[\mu,\sigma^2]$ &
  Normal-scaled Inverse Gamma($\lambda,\nu,\alpha,\beta$) &
  $\displaystyle\frac{\nu\lambda+n\xbar}{\nu+n}$, \qquad $\nu+n$, \qquad
  $\displaystyle \alpha + \frac{n}{2}$, \qquad
  $\displaystyle \beta + \frac{1}{2}\sumin(x_i-\xbar)^2 +
  \frac{\gamma(\xbar-\lambda)^2}{2(n+\gamma)}$
  \\[4ex]

  MVN($\mu,\Sigma_c$) & MVN($\mu_0,\Sigma_0$) &
  $\displaystyle\left( \Sigma_0^{-1}+ n\Sigma_c^{-1} \right)^{-1}
  \left( \Sigma_0^{-1}\mu_0 + n\Sigma^{-1} \xbar \right)$,
  $\displaystyle\left( \Sigma_0^{-1} + n \Sigma_c^{-1} \right)^{-1}$ \\[1ex]

  MVN($\mu_c,\Sigma$) & Inverse-Wishart($\kappa,\Psi$) &
  $\displaystyle n + \kappa, \Psi + \sumin(x_i-\mu_c)(x_i-\mu_c)^T$\\

  Pareto($x_{m_c}, k$) & $\gam[\alpha,\beta]$ &
  $\displaystyle\alpha+n, \beta + \sumin \log\frac{x_i}{x_{m_c}}$ \\

  Pareto($x_m, k_c$) & Pareto($x_0, k_0$) &
  $\displaystyle x_0, k_0 - kn$ where $k_0 > kn$ \\

  $\gam[\alpha_c,\beta]$ & $\gam[\alpha_0,\beta_0]$ &
  $\displaystyle \alpha_0 + n\alpha_c, \beta_0 + \sumin x_i$ \\[3ex]

  \hline
\end{tabular}

\vfill~
\columnbreak

\begin{tabular}{*3{|>{\begin{math}\displaystyle}l<{\end{math}}}|}
  \hline
  \multicolumn{3}{|c|}{Discrete likelihood}\\
  \hline && \\[-2ex]
  \text{Likelihood} & \text{Conjugate prior} &
    \text{Posterior hyperparameters} \\[1ex]

  \hline && \\[-2ex]

  \bern[p] & \bet[\alpha,\beta] &
  \alpha + \sumin x_i, \beta + n - \sumin x_i \\

  \bin[p] & \bet[\alpha,\beta] &
  \alpha + \sumin x_i, \beta + \sumin N_i - \sumin x_i \\

  \nbin[p] & \bet[\alpha,\beta] & \alpha + rn, \beta + \sumin x_i \\

  \pois[\lambda] & \gam[\alpha,\beta] & \alpha + \sumin x_i, \beta + n \\

  \text{Multinomial}(p) & \dir[\alpha] & \alpha + \sumin x^{(i)} \\

  \geo[p] & \bet[\alpha,\beta] & \alpha + n, \beta + \sumin x_i \\[3ex]

  \hline
\end{tabular}

\subsection{Bayesian Testing}
If $H_0:\theta \in \Theta_0$:
\begin{align*}
  \text{Prior probability } \Pr{H_0}
    &= \int_{\Theta_0} f(\theta)\,d\theta\\
  \text{Posterior probability } \Pr{H_0\giv x^n}
    &= \int_{\Theta_0} f(\theta\giv x^n)\,d\theta\\
\end{align*}

Let $H_0{\ldots}H_{k-1}$ be $k$ hypotheses.
Suppose $\theta \dist f(\theta\giv H_k)$,
\[\Pr{H_k\giv x^n}
= \frac{f(x^n\giv H_k)\Pr{H_k}}{\sum_{k=1}^K f(x^n\giv H_k)\Pr{H_k}},\]

Marginal likelihood
\[f(x^n\giv H_i)
= \int_\Theta f(x^n\giv \theta,H_i)f(\theta\giv H_i) \,d\theta\]

Posterior odds (of $H_i$ relative to $H_j$)
\[\frac{\Pr{H_i\giv x^n}}{\Pr{H_j\giv x^n}} \quad
= \underbrace{\frac{f(x^n\giv H_i)}{f(x^n\giv H_j)}}%
  _{\text{Bayes Factor }BF_{ij}}
\times \;\underbrace{\frac{\Pr{H_i}}{\Pr{H_j}}}_{\text{prior odds}}\]

\columnbreak
Bayes factor

\centering
\begin{tabular}{lll}
  $\log_{10} BF_{10}$ & $BF_{10}$ & \textsf{evidence} \\
  \toprule
  $0 - 0.5$ & $1 - 1.5$   & Weak \\
  $0.5 - 1$ & $1.5 - 10$  & Moderate \\
  $1 - 2$   & $10 - 100$  & Strong \\
  $> 2$     & $> 100$     & Decisive \\
\end{tabular}

\vspace*{2ex}

$p^* = \displaystyle\frac{\frac{p}{1-p}BF_{10}}{1+\frac{p}{1-p}BF_{10}}$
where $p=\Pr{H_1}$ and $p^* = \Pr{H_1 \giv x^n}$

\raggedright

\section{Sampling Methods}

\subsection{Inverse Transform Sampling}

Setup
\begin{itemize}
  \item $U \dist \unif[0,1]$
  \item $X \dist F$
  \item $F^{-1}(u)= \inf\{ x \mid F(x) \ge u\}$
\end{itemize}

Algorithm
\begin{enumerate}
  \item Generate $u \dist \unif[0,1]$
  \item Compute $x = F^{-1}(u)$
\end{enumerate}

\subsection{The Bootstrap}

Let $T_n = g(X_1,\dots,X_n)$ be a statistic.
\begin{enumerate}
  \item Estimate $\V[F]{T_n}$ with $\V[\Fnhat]{T_n}$.
  \item Approximate $\V[\Fnhat]{T_n}$ using simulation:
    \begin{enumerate}
      \item Repeat the following $B$ times to get $T_{n,1}^*,\dots,T_{n,B}^*$,
        an \iid sample from the sampling distribution implied by $\Fnhat$
        \begin{enumerate}
          \item Sample uniformly $X_1^*,\dots,X_n^* \dist \Fnhat$.
          \item Compute $T_n^* = g(X_1^*,\dots,X_n^*)$.
        \end{enumerate}
      \item Then
        \[v_{boot} = \widehat{\mathbb{V}}_{\Fnhat} =
          \displaystyle\frac{1}{B} \sum_{b=1}^B
            \left(T_{n,b}^* - \frac{1}{B} \sum_{r=1}^B T_{n,r}^* \right)^2\]
    \end{enumerate}
\end{enumerate}

\subsubsection{Bootstrap Confidence Intervals}

Normal-based interval
\[T_n \pm \zat\sehat_{boot}\]

Pivotal interval
\begin{enumerate}
  \item Location parameter $\theta = T(F)$
  \item Pivot $R_n = \that_n - \theta$
  \item Let $H(r) = \Pr{R_n \le r}$ be the \cdf of $R_n$
  \item Let $R_{n,b}^* = \that_{n,b}^* - \that_n$.
    Approximate $H$ using bootstrap:
    \[\widehat{H}(r) = \frac{1}{B}\sum_{b=1}^B I(R_{n,b}^* \le r)\]
  \item $\theta_\beta^*$ = $\beta$ sample quantile of
    $(\that_{n,1}^*,\dots,\that_{n,B}^*)$
  \item $r_\beta^*$ = beta sample quantile of
    $(R_{n,1}^*,\dots,R_{n,B}^*)$, i.e.,
    $r_\beta^* = \theta_\beta^* - \that_n$
  \item Approximate $1-\alpha$ confidence interval
    $C_n = \left( \hat{a}, \hat{b} \right)$ where
    \begin{align*}
    \hat{a}
      &=& \that_n - \widehat{H}^{-1}\left( 1-\frac{\alpha}{2} \right)
      &=& \that_n - r_{1-\alpha/2}^*
      &=& 2\that_n - \theta_{1-\alpha/2}^* \\
    \hat{b}
      &=& \that_n - \widehat{H}^{-1}\left( \frac{\alpha}{2} \right)
      &=& \that_n - r_{\alpha/2}^*
      &=& 2\that_n - \theta_{\alpha/2}^*
    \end{align*}
\end{enumerate}
%  \[C_n = \left( 2\that_n - \that_{1-\alpha/2}^*,
%    2\that_n + \that_{\alpha/2}^* \right)\]

Percentile interval
\[C_n = \left( \theta_{\alpha/2}^*, \theta_{1-\alpha/2}^* \right)\]

\subsection{Rejection Sampling}
Setup
\begin{itemize}
  \item We can easily sample from $g(\theta)$
  \item We want to sample from $h(\theta)$, but it is difficult
  \item We know $h(\theta)$ up to a proportional constant:
    $h(\theta) = \displaystyle\frac{k(\theta)}{\int k(\theta)\,d\theta}$
  \item Envelope condition: we can find $M > 0$ such that
    $k(\theta) \le Mg(\theta) \quad \forall \theta$
\end{itemize}

Algorithm
\begin{enumerate}
  \item Draw $\theta^{cand} \dist g(\theta)$
  \item Generate $u \dist \unif[0,1]$
  \item Accept $\theta^{cand}$ if
    $u \le \displaystyle\frac{k(\theta^{cand})}{Mg(\theta^{cand})}$
  \item Repeat until $B$ values of $\theta^{cand}$ have been accepted
\end{enumerate}

Example
\begin{itemize}
  \item We can easily sample from the prior $g(\theta) = f(\theta)$
  \item Target is the posterior
    $h(\theta) \propto k(\theta) = f(x^n\giv \theta) f(\theta)$
  \item Envelope condition:
    $f(x^n\giv\theta) \le f(x^n\giv\that_n) = \Lln(\that_n)\equiv M$
  \item Algorithm
    \begin{enumerate}
      \item Draw $\theta^{cand} \dist f(\theta)$
      \item Generate $u \dist \unif[0,1]$
      \item Accept $\theta^{cand}$ if
        $u \le \displaystyle\frac{\Lln(\theta^{cand})}{\Lln(\that_n)}$
    \end{enumerate}
\end{itemize}

\subsection{Importance Sampling}

Sample from an importance function $g$ rather than target density $h$.\\
Algorithm to obtain an approximation to $\E{q(\theta) \giv x^n}$:
\begin{enumerate}
  \item Sample from the prior $\theta_1,\ldots,\theta_n \distiid f(\theta)$
  \item $w_i = \displaystyle\frac{\Lln(\theta_i)}{\sum_{i=1}^B \Lln(\theta_i)}
    \quad\forall i = 1,\ldots,B$
  \item $\E{q(\theta)\giv x^n} \approx \sum_{i=1}^B q(\theta_i)w_i$
\end{enumerate}

\section{Decision Theory}

Definitions
\begin{itemize}
  \item Unknown quantity affecting our decision: $\theta \in \Theta$
  \item Decision rule: synonymous for an estimator $\that$
  \item Action $a \in \mathcal{A}$: possible value of the decision rule. In the
    estimation context, the action is just an estimate of $\theta$, $\that(x)$.
  \item Loss function $L$: consequences of taking action $a$ when true state is
    $\theta$ or discrepancy between $\theta$ and $\that$,
    $L: \Theta \times \mathcal{A} \to [-k,\infty).$
\end{itemize}

Loss functions
\begin{itemize}
  \item Squared error loss: $L(\theta,a) = (\theta-a)^2$
  \item Linear loss: $L(\theta,a) = \begin{cases}
      K_1(\theta-a) & a-\theta < 0 \\
      K_2(a-\theta) & a-\theta \ge 0
    \end{cases}$
  \item Absolute error loss: $L(\theta,a) = |\theta-a| \quad$
    (linear loss with $K_1=K_2$)
  \item $L_p$ loss: $L(\theta,a) = |\theta-a|^p$
  \item Zero-one loss: $L(\theta,a) = \begin{cases}
      0 & a=\theta \\
      1 & a\neq\theta \\
    \end{cases}$
\end{itemize}

\subsection{Risk}

Posterior risk
\[r(\that \giv x)
= \int L(\theta,\that(x))f(\theta\giv x)\,d\theta
= \E[\theta|X]{L(\theta,\that(x))}\]

(Frequentist) risk
\[R(\theta,\that)
= \int L(\theta,\that(x)) f(x\giv\theta) \dx
= \E[X|\theta]{L(\theta,\that(X))}\]

Bayes risk
\[r(f,\that)
= \iint L(\theta,\that(x))f(x,\theta)\dx\dtheta
= \E[\theta,X]{L(\theta,\that(X))}\]
\[r(f,\that)
= \E[\theta]{\E[X|\theta]{L(\theta,\that(X)}}
= \E[\theta]{R(\theta,\that)}\]
\[r(f,\that)
= \E[X]{\E[\theta|X]{L(\theta,\that(X)}}
= \E[X]{r(\that\giv X)}\]

\subsection{Admissibility}

\begin{itemize}
  \item $\that'$ dominates $\that$ if
    \[\forall \theta: R(\theta,\that') \le R(\theta,\that)\]
    \[\exists \theta: R(\theta,\that') < R(\theta,\that)\]
  \item $\that$ is inadmissible if there is at least one other estimator
    $\that'$ that dominates it. Otherwise it is called admissible.
\end{itemize}

\subsection{Bayes Rule}

Bayes rule (or Bayes estimator)
\begin{itemize}
  \item $r(f,\that) = \inf_{\ttil} r(f,\ttil)$
  \item $\that(x) = \inf r(\that\giv x) \; \forall x
\imp r(f,\that) = \int r(\that\giv x)f(x)\,dx$
\end{itemize}

Theorems
\begin{itemize}
  \item Squared error loss: posterior mean
  \item Absolute error loss: posterior median
  \item Zero-one loss: posterior mode
\end{itemize}

\subsection{Minimax Rules}

Maximum risk
\[\bar{R}(\that) = \sup_\theta R(\theta,\that) \qquad
\bar{R}(a) = \sup_\theta R(\theta,a)\]

Minimax rule
\[\sup_\theta R(\theta,\that)
= \inf_{\ttil} \bar{R}(\ttil)
= \inf_{\ttil} \sup_\theta R(\theta,\ttil)\]

\[\that = \text{Bayes rule} \; \wedge \;
\exists c: R(\theta,\that) = c\]

Least favorable prior
\[\that^f = \text{Bayes rule} \; \wedge \;
R(\theta,\that^f) \le r(f,\that^f) \;\forall\theta\]

\section{Linear Regression}

Definitions
\begin{itemize}
  \item Response variable $Y$
  \item Covariate $X$ (aka predictor variable or feature)
\end{itemize}

\subsection{Simple Linear Regression}
Model
\[Y_i = \beta_0 + \beta_1 X_i + \epsilon_i
\qquad \E{\epsilon_i\giv X_i} = 0 ,\; \V{\epsilon_i\giv X_i} = \sigma^2\]

Fitted line
\[\rhat(x) = \bhat_0 + \bhat_1 x\]

Predicted (fitted) values
\[\Yhat_i = \rhat(X_i)\]

Residuals
\[\ehat_i = Y_i - \Yhat_i
= Y_i - \left( \bhat_0 + \bhat_1 X_i \right)\]

Residual sums of squares (\rss)
\[\rss(\bhat_0,\bhat_1) = \sumin \ehat_i^2\]

Least square estimates
\[\bhat^T=(\bhat_0, \bhat_1)^T: \min_{\bhat_0,\bhat_1}\rss\]
\begin{align*}
  \bhat_0 &= \bar Y_n - \bhat_1 \bar X_n \\
  \bhat_1 &= \frac{\sumin(X_i-\bar X_n)(Y_i-\bar Y_n)}{\sumin(X_i - \bar X_n)^2}
           = \frac{\sumin X_iY_i-n\Xbar\Ybar}{\sumin X_i^2 - n\Xsqbar} \\
  \E{\bhat\giv X^n} &= \begin{pmatrix}\beta_0 \\ \beta_1\end{pmatrix} \\
  \V{\bhat\giv X^n} &=
  \frac{\sigma^2}{n s^2_X}
    \begin{pmatrix}n^{-1}\sumin X_i^2 & -\Xnbar \\ -\Xnbar & 1\end{pmatrix} \\
  \sehat(\bhat_0) &= \frac{\shat}{s_X\sqrt{n}} \sqrt{\frac{\sumin X_i^2}{n}} \\
  \sehat(\bhat_1) &= \frac{\shat}{s_X\sqrt{n}}
\end{align*}
where $s_X^2 = n^{-1} \sumin(X_i-\Xnbar)^2$ and $\shat^2 =
\frac{1}{n-2} \sumin \ehat_i^2$ (unbiased estimate).

Further properties:
\begin{itemize}
  \item Consistency:
    $\bhat_0 \pconv \beta_0$ and $\bhat_1 \pconv \beta_1$
  \item Asymptotic normality:
    \[\frac{\bhat_0 - \beta_0}{\sehat(\bhat_0)} \dconv \norm[0,1]
    \quad\text{and}\quad
    \frac{\bhat_1 - \beta_1}{\sehat(\bhat_1)} \dconv \norm[0,1]\]
  \item Approximate $1-\alpha$ confidence intervals for $\beta_0$ and $\beta_1$:
    \[\bhat_0 \pm \zat \sehat(\bhat_0) \quad\text{and}\quad
    \bhat_1 \pm \zat \sehat(\bhat_1)\]
  \item Wald test for \hyp{\beta_1=0}{\beta_1\neq 0}: reject
    $H_0$ if $|W| > \zat$ where $W = \bhat_1/\sehat(\bhat_1)$.
\end{itemize}

R$^2$
\[R^2
= \frac{\sumin(\Yhat_i-\Ybar)^2}{\sumin(Y_i-\Ybar)^2}
= 1 - \frac{\sumin \ehat_i^2}{\sumin(Y_i-\Ybar)^2}
= 1 - \frac{\rss}{\tss}\]

Likelihood
\begin{align*}
\Ll &= \prodin f(X_i,Y_i)
= \prodin f_X(X_i) \times \prodin f_{Y|X}(Y_i \giv X_i) = \Ll_1 \times \Ll_2 \\
\Ll_1 &= \prodin f_X(X_i) \\
\Ll_2 &= \prodin f_{Y|X}(Y_i \giv X_i)
\propto \sigma^{-n}
\Exp{-\frac{1}{2\sigma^2}\sum_i\Bigl(Y_i-(\beta_0-\beta_1X_i)\Bigr)^2}
\end{align*}

Under the assumption of Normality, the least squares estimator is
also the \mle but the least squares variance estimator is not the \mle.
\[\shat^2 = \frac{1}{n}\sumin \ehat_i^2\]

\subsection{Prediction}

Observe $X = x_*$ of the covariate and want to predict their outcome $Y_*$.
\begin{align*}
\Yhat_* &= \bhat_0 + \bhat_1 x_* \\
\V{\Yhat_*} &= \V{\bhat_0} + x_*^2 \V{\bhat_1} + 2x_* \cov{\bhat_0,\bhat_1}
\end{align*}

Prediction interval
\[\xihat_n^2
= \shat^2\left( \frac{\sumin(X_i-X_*)^2}{n\sum_i(X_i-\Xbar)^2j}+1 \right)\]
\[\Yhat_* \pm \zat \xihat_n\]

\subsection{Multiple Regression}

\[Y = X\beta + \epsilon\]
where
\[X =
\begin{pmatrix} X_{11} & \cdots & X_{1k} \\ \vdots & \ddots & \vdots \\
  X_{n1} & \cdots & X_{nk}\end{pmatrix} \quad
\beta = \begin{pmatrix}\beta_1 \\ \vdots \\ \beta_k\end{pmatrix} \quad
\epsilon = \begin{pmatrix}\epsilon_1 \\ \vdots \\ \epsilon_n\end{pmatrix}\]

Likelihood
\[\Ll(\mu,\Sigma) = (2\pi\sigma^2)^{-n/2} \Exp{-\frac{1}{2\sigma^2}\rss}\]
\[\rss = (y-X\beta)^T(y-X\beta) = \|Y-X\beta\|^2 = \sumiN(Y_i-x_i^T\beta)^2\]

If the $(k \times k)$ matrix $X^TX$ is invertible,
\begin{align*}
  \bhat &= (X^TX)^{-1}X^TY \\
  \V{\bhat \giv X^n} &= \sigma^2(X^TX)^{-1} \\
  \bhat &\approx \norm[\beta, \sigma^2(X^TX)^{-1}]
\end{align*}

Estimate regression function
\[\rhat(x) = \sumjk\bhat_j x_j\]

Unbiased estimate for $\sigma^2$
\[\shat^2 = \frac{1}{n-k} \sumin \ehat_i^2 \qquad \ehat = X\bhat-Y\]

\mle
\[\mhat = \Xbar \qquad \shat^2 = \frac{n-k}{n}\sigma^2\]

$1-\alpha$ Confidence interval
\[\bhat_j \pm \zat\sehat(\bhat_j)\]

\subsection{Model Selection}

Consider predicting a new observation $Y^*$ for covariates $X^*$ and let $S
\subset J$ denote a subset of the covariates in the model, where $|S| = k$ and
$|J| = n$.

Issues
\begin{itemize}
  \item Underfitting: too few covariates yields high bias
  \item Overfitting: too many covariates yields high variance
\end{itemize}

Procedure
\begin{enumerate}
  \item Assign a score to each model
  \item Search through all models to find the one with the highest score
\end{enumerate}

Hypothesis testing
\[\hyp{\beta_j=0}{\beta_j\neq0} \quad\forall j \in J\]

Mean squared prediction error (\mspe)
\[\mspe = \E{(\Yhat(S)-Y^*)^2}\]

Prediction risk
\[R(S) = \sumin \mspe_i = \sumin \E{(\Yhat_i(S)-Y_i^*)^2}\]

Training error
\[\Rhat_{tr}(S) = \sumin(\Yhat_i(S)-Y_i)^2\]

$R^2$
\[R^2(S)
= 1 - \frac{\rss(S)}{\tss}
= 1 - \frac{\Rhat_{tr}(S)}{\tss}
= 1 - \frac{\sumin(\Yhat_i(S)-\Ybar)^2}{\sumin(Y_i-\Ybar)^2}\]

The training error is a downward-biased estimate of the prediction risk.
\[\E{\Rhat_{tr}(S)} < R(S)\]
\[\bias(\Rhat_{tr}(S)) = \E{\Rhat_{tr}(S)} - R(S) = -2\sumin\cov{\Yhat_i,Y_i}\]

Adjusted $R^2$
\[R^2(S) = 1 - \frac{n-1}{n-k} \frac{\rss}{\tss}\]

\textsc{Mallow's} $C_p$ statistic
\[\Rhat(S) = \Rhat_{tr}(S) + 2k\shat^2
= \text{lack of fit} + \text{complexity penalty}\]

\textsc{Akaike} Information Criterion (AIC)
\[AIC(S) = \lln(\bhat_S, \shat^2_S) - k\]

Bayesian Information Criterion (BIC)
\[BIC(S) = \lln(\bhat_S, \shat^2_S) - \frac{k}{2}\log n\]

Validation and training
\[\Rhat_V(S) = \sumim(\Yhat_i^*(S) - Y_i^*)^2 \qquad
m = |\{\text{validation data}\}|,
\text{ often }\frac{n}{4}\text { or }\frac{n}{2}\]

Leave-one-out cross-validation
\[\Rhat_{CV}(S)
= \sumin(Y_i - \Yhat_{(i)})^2
= \sumin \left( \frac{Y_i-\Yhat_i(S)}{1-U_{ii}(S)} \right)^2\]
\[U(S) = X_S(X_S^T X_S)^{-1} X_S \text{ (``hat matrix'')}\]

\section{Non-parametric Function Estimation}

\subsection{Density Estimation}

Estimate $f(x)$, where $f(x) = \Pr{X \in A} = \int_A f(x)\dx$.\\

Integrated square error (\ise)
\[L(f, \fnhat) = \int\left(f(x) - \fnhat(x)\right)^2 \dx = J(h)+\int f^2(x)\dx\]

Frequentist risk
\[R(f, \fnhat) = \E{L(f,\fnhat)} = \int b^2(x) \dx + \int v(x) \dx\]
\begin{align*}
  b(x) &= \E{\fnhat(x)} - f(x) \\
  v(x) &= \V{\fnhat(x)}
\end{align*}

\subsubsection{Histograms}

Definitions
\begin{itemize}
  \item Number of bins $m$
  \item Binwidth $h = \frac{1}{m}$
  \item Bin $B_j$ has $\nu_j$ observations
  \item Define $\phat_j = \nu_j/n$ and $p_j = \int_{B_j} f(u)\du$
\end{itemize}

Histogram estimator
\begin{align*}
\fnhat(x) &= \sumjm \frac{\phat_j}{h} I(x\in B_j) \\
\E{\fnhat(x)} &= \frac{p_j}{h} \\
\V{\fnhat(x)} &= \frac{p_j(1-p_j)}{nh^2} \\
R(\fnhat,f) &\approx
\frac{h^2}{12} \int \left(f'(u)\right)^2 \du + \frac{1}{nh} \\
h^* &= \frac{1}{n^{1/3}} \left( \frac{6}{\int\left(f'(u) \right)^2}\du
\right)^{1/3} \\
R^*(\fnhat,f) &\approx \frac{C}{n^{2/3}} \qquad
C = \left(\frac{3}{4}\right)^{2/3} \left( \int\left( f'(u) \right)^2 \du
\right)^{1/3}
\end{align*}

Cross-validation estimate of $\E{J(h)}$
\[\Jhat_{CV}(h)
= \int \fnhat^2(x) \dx - \frac{2}{n}\sumin \fhat_{(-i)}(X_i)
= \frac{2}{(n-1)h} - \frac{n+1}{(n-1)h} \sumjm \phat_j^2\]

\subsubsection{Kernel Density Estimator (KDE)}

Kernel $K$
\begin{itemize}
  \item $K(x) \ge 0$
  \item $\int K(x)\dx = 1$
  \item $\int xK(x)\dx = 0$
  \item $\int x^2 K(x)\dx \equiv \sigma^2_K > 0$
\end{itemize}

KDE
\begin{align*}
  \fnhat(x) &= \frac{1}{n} \sumin \frac{1}{h} K\left( \frac{x-X_i}{h} \right) \\
  R(f,\fnhat) &\approx \frac{1}{4}(h\sigma_K)^4 \int (f''(x))^2\dx
  + \frac{1}{nh} \int K^2(x)\dx \\
  h^* &= \frac{c_1^{-2/5} c_2^{-1/5} c_3^{-1/5}}{n^{1/5}} \qquad
  c_1=\sigma_K^2,\;c_2 = \int K^2(x)\dx,\;c_3 = \int(f''(x))^2\dx\\
  R^*(f,\fnhat) &= \frac{c_4}{n^{4/5}} \qquad
  c_4 = \underbrace{\frac{5}{4}(\sigma_K^2)^{2/5} \left(\int
  K^2(x)\dx\right)^{4/5}}_{C(K)}
  \left( \int(f'')^2\dx \right)^{1/5}
\end{align*}

\textsc{Epanechnikov} Kernel
\[K(x) = \begin{cases}
  \frac{3}{4\sqrt{5}(1-x^2/5)} & |x| < \sqrt{5} \\ 0 & \text{otherwise}
\end{cases}\]

Cross-validation estimate of $\E{J(h)}$
\[\Jhat_{CV}(h)
= \int \fnhat^2(x) \dx - \frac{2}{n}\sumin \fhat_{(-i)}(X_i)
\approx \frac{1}{hn^2} \sumin \sumjn K^*\left( \frac{X_i-X_j}{h} \right) +
\frac{2}{nh} K(0)\]
\[K^*(x) = K^{(2)}(x)-2K(x) \qquad K^{(2)}(x) = \int K(x-y) K(y) \dy\]

\subsection{Non-parametric Regression}

Estimate $f(x)$ where $f(x) = \E{Y \giv X=x}$.
Consider pairs of points $(x_1,Y_1),\dots,(x_n,Y_n)$ related by
\begin{align*}
Y_i &= r(x_i) + \epsilon_i \\
\E{\epsilon_i} &= 0 \\
\V{\epsilon_i} &= \sigma^2
\end{align*}

$k$-nearest Neighbor Estimator
\[\rhat(x) = \frac{1}{k} \sum_{i:x_i \in N_k(x)} Y_i \qquad \text{where }
N_k(x) = \{k \text{ values of } x_1,\dots,x_n \text{ closest to } x\}\]

\textsc{Nadaraya-Watson} Kernel Estimator
\begin{align*}
\rhat(x) &= \sumin w_i(x)Y_i \\
w_i(x)
&= \frac{K\left(\frac{x-x_i}{h}\right)}{\sumjn K\left(\frac{x-x_j}{h}\right)}
\quad \in [0,1] \\
R(\rhat_n,r) &\approx \frac{h^4}{4} \left( \int x^2K^2(x)\dx \right)^4
\int \left( r''(x) + 2r'(x)\frac{f'(x)}{f(x)}\right)^2 \dx \\
&+ \int \frac{\sigma^2 \int K^2(x) \dx}{nhf(x)}\dx \\
h^* &\approx \frac{c_1}{n^{1/5}} \\
R^*(\rhat_n,r) &\approx \frac{c_2}{n^{4/5}} \\
\end{align*}

Cross-validation estimate of $\E{J(h)}$
\[\Jhat_{CV}(h)
= \sumin (Y_i - \rhat_{(-i)}(x_i))^2
= \sumin \frac{(Y_i - \rhat(x_i))^2}{\left(
1- \frac{K(0)}{\sumjn K\left(\frac{x-x_j}{h}\right)}\right)^2}\]

\subsection{Smoothing Using Orthogonal Functions}

Approximation
\[r(x) = \sum_{j=1}^\infty\beta_j\phi_j(x)
\approx \sum_{j=1}^J \beta_j\phi_j(x)\]

Multivariate regression
\[Y = \Phi\beta + \eta\]
\[\text{where}\quad \eta_i = \epsilon_i \quad\text{and}\quad \Phi
= \begin{pmatrix}
  \phi_0(x_1) & \cdots & \phi_J(x_1) \\
  \vdots & \ddots & \vdots \\
  \phi_0(x_n) & \cdots & \phi_J(x_n)
  \end{pmatrix}\]

Least squares estimator
\begin{align*}
\bhat &= (\Phi^T\Phi)^{-1}\Phi^T Y \\
&\approx \frac{1}{n}\Phi^T Y
  \quad\text{(for equally spaced observations only)}
\end{align*}

Cross-validation estimate of $\E{J(h)}$
\[\Rhat_{CV}(J)
= \sumin \left( Y_i - \sum_{j=1}^J \phi_j(x_i)\bhat_{j,(-i)} \right)^2\]

\section{Stochastic Processes}

Stochastic Process
\[\left\{ X_t : t \in T\right\} \qquad T=\begin{cases}\{0,\pm1,\dots\}=\Z &
  \text{discrete} \\ [0,\infty) & \text{continuous}\end{cases}\]

\begin{itemize}
  \item Notations $X_t$, $X(t)$
  \item State space $\mathcal{X}$
  \item Index set $T$
\end{itemize}

\subsection{Markov Chains}

Markov chain
\[\Pr{X_n = x \giv X_0,\dots,X_{n-1}} = \Pr{X_n = x \giv X_{n-1}}
\quad \forall n\in T, x \in \mathcal{X}\]

Transition probabilities
\begin{align*}
p_{ij} &\equiv \Pr{X_{n+1} = j \giv X_n = i} \\
p_{ij}(n) &\equiv \Pr{X_{m+n} = j \giv X_m = i} \quad\text{n-step}
\end{align*}

Transition matrix $\mathbf{P}$ (n-step: $\mathbf{P}_n$)
\begin{itemize}
  \item $(i,j)$ element is $p_{ij}$
  \item $p_{ij} > 0$
  \item $\sum_i p_{ij} = 1$
\end{itemize}

\textsc{Chapman-Kolmogorov}
\[p_{ij}(m+n) = \sum_k p_{ij}(m) p_{kj}(n)\]
\[\mathbf{P}_{m+n} = \mathbf{P}_m\mathbf{P}_n\]
\[\mathbf{P}_n = \mathbf{P} \times \cdots \times \mathbf{P} = \mathbf{P}^n\]

Marginal probability
\begin{align*}
  \mu_n &= (\mu_n(1),\dots,\mu_n(N))
  \quad\text{where}\quad \mu_i(i)=\Pr{X_n=i} \\
\mu_0 &\eqdef \text{initial distribution} \\
\mu_n &= \mu_0\mathbf{P}^n
\end{align*}

\subsection{Poisson Processes}

Poisson process
\begin{itemize}
  \item $\left\{ X_t : t \in [0,\infty) \right\}$
    = number of events up to and including time $t$
  \item $X_0 = 0$
  \item Independent increments:
    \[\forall t_0 < \cdots < t_n:
    X_{t_1} - X_{t_0} \ind \cdots \ind X_{t_n} - X_{t_{n-1}}\]
  \item Intensity function $\lambda(t)$
    \begin{itemize}
      \item $\Pr{X_{t+h}-X_t = 1} = \lambda(t) h + o(h)$
      \item $\Pr{X_{t+h}-X_t = 2} = o(h)$
    \end{itemize}
  \item $X_{s+t} - X_s \dist \pois[m(s+t)-m(s)]\;$ where
    $\;m(t)=\int_0^t\lambda(s)\ds$
\end{itemize}

Homogeneous Poisson process
\[\lambda(t) \equiv \lambda \imp X_t \dist \pois[\lambda t] \qquad \lambda > 0\]

Waiting times
\[W_t \define \text{time at which $X_t$ occurs}\]
\[W_t \dist \gam[t, \frac{1}{\lambda}]\]

Interarrival times
\[S_t = W_{t+1} - W_t\]
\[S_t \dist \ex[\frac{1}{\lambda}]\]

\begin{center}
  \begin{tikzpicture}[decoration={brace,amplitude=5pt}]
    \draw[->] (0,0) -- (8,0) node[below]{$t$};
    \foreach \i in {1,1.5,3,5,6,7}
      \draw (\i,2pt) -- (\i,-2pt) node {};
    \draw (3,0) node[below] {\footnotesize $W_{t-1}$};
    \draw (5,0) node[below] {\footnotesize $W_{t}$};
    \draw[decorate,yshift=5pt] (3,0) -- (5,0)
      node[midway,above=3pt] {\footnotesize $S_t$};
  \end{tikzpicture}
\end{center}

\section{Time Series}

Mean function
\[\mu_{x_t} = \E{x_t} = \int_{-\infty}^\infty x f_t(x) \dx\]

Autocovariance function
\[\gamma_x(s,t) = \E{(x_s-\mu_s)(x_t-\mu_t)} = \E{x_sx_t} - \mu_s\mu_t\]
\[\gamma_x(t,t) = \E{(x_t-\mu_t)^2} = \V{x_t}\]

Autocorrelation function (ACF)
\[\rho(s,t) = \frac{\cov{x_s,x_t}}{\sqrt{\V{x_s}\V{x_t}}}
            = \frac{\gamma(s,t)}{\sqrt{\gamma(s,s)\gamma(t,t)}}\]

Cross-covariance function (CCV)
\[\gamma_{xy}(s,t) = \E{(x_s-\mu_{x_s})(y_t-\mu_{y_t})}\]

Cross-correlation function (CCF)
\[\rho_{xy}(s,t) = \frac{\gamma_{xy}(s,t)}{\sqrt{\gamma_x(s,s)\gamma_y(t,t)}}\]

Backshift operator
\[B^k(x_t) = x_{t-k}\]

Difference operator
\[\nabla^d = (1-B)^d\]

White noise
\begin{itemize}
  \item $w_t \dist wn(0, \sigma_w^2)$
  \item Gaussian: $w_t \distiid \norm[0, \sigma_w^2]$
  \item $\E{w_t} = 0 \quad t\in T$
  \item $\V{w_t} = \sigma^2 \quad t\in T$
  \item $\gamma_w(s,t) = 0 \quad s \neq t \;\wedge\; s,t\in T$
\end{itemize}

%Auto regression
%\[x_t = \sum_{i=1}^p \phi_i x_{t-i} + w_t\]

Random walk
\begin{itemize}
  \item Drift $\delta$
  \item $x_t = \delta t + \sum_{j=1}^t w_j$
  \item $\E{x_t} = \delta t$
\end{itemize}

Symmetric moving average
\[m_t = \sum_{j=-k}^k a_j x_{t-j}
\qquad \text{where } a_j=a_{-j}\ge0 \text{ and } \sum_{j=-k}^k a_j = 1\]

\subsection{Stationary Time Series}

Strictly stationary
\[\Pr{x_{t_1} \le c_1, \dots, x_{t_k} \le c_k} =
  \Pr{x_{t_1+h} \le c_1, \dots, x_{t_k+h} \le c_k}\]
\[\forall k\in\N,t_k, c_k, h\in\Z\]

Weakly stationary
\begin{itemize}
  \item $\E{x_t^2} < \infty \qquad\forall t\in\Z$
  \item $\E{x_t^2} = m \qquad\forall t\in\Z$
  \item $\gamma_x(s,t) = \gamma_x(s+r, t+r) \qquad\forall r,s,t\in\Z$
\end{itemize}

Autocovariance function
\begin{itemize}
  \item $\gamma(h) = \E{(x_{t+h}-\mu)(x_t-\mu)} \qquad \forall h\in\Z$
  \item $\gamma(0) = \E{(x_t-\mu)^2}$
  \item $\gamma(0) \ge 0$
  \item $\gamma(0) \ge |\gamma(h)|$
  \item $\gamma(h) = \gamma(-h)$
\end{itemize}

Autocorrelation function (ACF)
\[\rho_x(h) = \frac{\cov{x_{t+h},x_t}}{\sqrt{\V{x_{t+h}}\V{x_t}}}
          = \frac{\gamma(t+h,t)}{\sqrt{\gamma(t+h,t+h)\gamma(t,t)}}
          = \frac{\gamma(h)}{\gamma(0)}\]

Jointly stationary time series
\[\gamma_{xy}(h) = \E{(x_{t+h}-\mu_x)(y_t-\mu_y)}\]
\[\rho_{xy}(h) = \frac{\gamma_{xy}(h)}{\sqrt{\gamma_x(0)\gamma_y(h)}}\]

Linear process
\[x_t = \mu + \sum_{j=-\infty}^\infty \psi_j w_{t-j} \quad\text{where}\quad
\sum_{j=-\infty}^\infty |\psi_j| < \infty\]
\[\gamma(h) = \sigma_w^2 \sum_{j=-\infty}^\infty \psi_{j+h}\psi_j\]

\subsection{Estimation of Correlation}

Sample mean
\[\xbar = \frac{1}{n}\sum_{t=1}^n x_t\]

Sample variance
\[\V{\xbar} = \frac{1}{n}\sum_{h=-n}^n \left(1-\frac{|h|}{n}\right)\gamma_x(h)\]

Sample autocovariance function
\[\ghat(h) = \frac{1}{n}\sum_{t=1}^{n-h}(x_{t+h}-\xbar)(x_t-\xbar)\]

Sample autocorrelation function
\[\rhohat(h) = \frac{\ghat(h)}{\ghat(0)}\]

Sample cross-variance function
\[\ghat_{xy}(h) = \frac{1}{n} \sum_{t=1}^{n-h}(x_{t+h}-\xbar)(y_t - \ybar)\]

Sample cross-correlation function
\[\rhohat_{xy}(h) = \frac{\ghat_{xy}(h)}{\sqrt{\ghat_x(0) \ghat_y(0)}}\]

Properties
\begin{itemize}
  \item $\sigma_{\rhohat_x(h)} = \displaystyle\frac{1}{\sqrt{n}}$
    if $x_t$ is white noise
  \item $\sigma_{\rhohat_{xy}(h)} = \displaystyle\frac{1}{\sqrt{n}}$
    if $x_t$ or $y_t$ is white noise
\end{itemize}

\subsection{Non-Stationary Time Series}

Classical decomposition model
\[x_t = \mu_t + s_t + w_t\]
\begin{itemize}
  \item $\mu_t =$ trend
  \item $s_t =$ seasonal component
  \item $w_t =$ random noise term
\end{itemize}

\subsubsection{Detrending}

Least squares
\begin{enumerate}
  \item Choose trend model, e.g.,
    $\mu_t = \beta_0 + \beta_1 t + \beta_2 t^2$
  \item Minimize \rss to obtain trend estimate
    $\mhat_t = \bhat_0 + \bhat_1 t + \bhat_2 t^2$
  \item Residuals $\triangleq$ noise $w_t$
\end{enumerate}

Moving average
\begin{itemize}
  \item The \emph{low-pass} filter $v_t$ is a symmetric moving average $m_t$
    with $a_j = \frac{1}{2k+1}$:
    \[v_t = \frac{1}{2k+1} \sum_{i=-k}^k x_{t-1}\]
  \item If $\frac{1}{2k+1} \sum_{i=-k}^k w_{t-j} \approx 0$,
    a linear trend function $\mu_t = \beta_0 + \beta_1t$ passes without
    distortion
\end{itemize}

Differencing
\begin{itemize}
  \item $\mu_t = \beta_0 + \beta_1t \imp \nabla x_t = \beta_1$
\end{itemize}

\subsection{ARIMA models}

Autoregressive polynomial
\[\phi(z) = 1 - \phi_1 z - \cdots - \phi_p z_p
\qquad z \in \C \wedge \phi_p \neq 0\]

Autoregressive operator
\[\phi(B) = 1 - \phi_1B - \cdots - \phi_pB^p\]

Autoregressive model order $p$, $\AR$
\[x_t = \phi_1 x_{t-1} + \cdots + \phi_p x_{t-p} + w_t \eqv \phi(B)x_t = w_t\]

$\AR[1]$
\begin{itemize}
  \item $x_t = \phi^k(x_{t-k}) + \displaystyle\sum_{j=0}^{k-1} \phi^j(w_{t-j})
    \stackrel{k\to\infty, |\phi| < 1}{=}
    \underbrace{\sum_{j=0}^\infty \phi^j(w_{t-j})}_{\text{linear process}}$
\item $\E{x_t} = \sum_{j=0}^\infty\phi^j(\E{w_{t-j}}) = 0$
\item $\gamma(h) = \cov{x_{t+h},x_t} = \frac{\sigma_w^2\phi^h}{1-\phi^2}$
\item $\rho(h) = \frac{\gamma(h)}{\gamma(0)} = \phi^h$
\item $\rho(h) = \phi \rho(h-1) \quad h=1,2,\ldots$
\end{itemize}

Moving average polynomial
\[\theta(z) = 1 + \theta_1 z + \cdots + \theta_q z_q
\qquad z \in \C \wedge \theta_q \neq 0\]

Moving average operator
\[\theta(B) = 1 + \theta_1B + \cdots + \theta_pB^p\]

$\MA$ (moving average model order $q$)
\[x_t = w_t + \theta_1 w_{t-1} + \cdots + \theta_q w_{t-q}
\eqv x_t = \theta(B)w_t\]
\[\E{x_t} = \sum_{j=0}^q \theta_j\E{w_{t-j}} = 0\]
\[\gamma(h) = \cov{x_{t+h},x_t} = \begin{cases}
  \sigma_w^2\sum_{j=0}^{q-h} \theta_j\theta_{j+h} & 0 \le h \le q \\
  0 & h > q
\end{cases}\]

$\MA[1]$
\[x_t = w_t + \theta w_{t-1}\]
\[\gamma(h) = \begin{cases}
  (1+\theta^2)\sigma_w^2 & h = 0 \\
  \theta\sigma_w^2 & h = 1 \\
  0 & h > 1
\end{cases}\]
\[\rho(h) = \begin{cases}
  \frac{\theta}{(1+\theta^2)} & h = 1 \\
  0 & h > 1
\end{cases}\]

$\ARMA$
\[x_t = \phi_1 x_{t-1} + \cdots + \phi_p x_{t-p} + w_t + \theta_1 w_{t-1} +
\cdots + \theta_q w_{t-q}\]
\[\phi(B) x_t = \theta(B) w_t\]

Partial autocorrelation function (PACF)
\begin{itemize}
  \item $x_i^{h-1} \eqdef$ regression of $x_i$ on
    $\{x_{h-1}, x_{h-2}, \dots, x_1\}$
  \item $\phi_{hh} = corr(x_h - x_h^{h-1}, x_0 - x_0^{h-1}) \quad h \ge 2$
  \item E.g., $\phi_{11} = corr(x_1,x_0) = \rho(1)$
\end{itemize}

$\ARIMA$
\[\nabla^d x_t = (1-B)^d x_t \text{ is } \ARMA\]
\[\phi(B)(1-B)^d x_t = \theta(B) w_t\]

Exponentially Weighted Moving Average (EWMA)
\[x_t = x_{t-1} + w_t - \lambda w_{t-1}\]
\[x_t = \sum_{j=1}^\infty(1-\lambda)\lambda^{j-1} x_{t-j} + w_t
\quad\text{when } |\lambda| < 1\]
\[\tilde{x}_{n+1} = (1-\lambda)x_n + \lambda \tilde{x}_n\]

\begin{titemize}{Seasonal ARIMA}
  \item Denoted by $\SARIMA$
  \item $\Phi_P(B^s) \phi(B) \nabla_s^D \nabla^d x_t
    = \delta + \Theta_Q(B^s)\theta(B)w_t$
\end{titemize}

\subsubsection{Causality and Invertibility}
$\ARMA$ is causal (future-independent)
$\eqv \exists \{\psi_j\} : \sum_{j=0}^\infty \psi_j < \infty$ such that
\[x_t = \sum_{j=0}^\infty w_{t-j} = \psi(B)w_t\]

$\ARMA$ is invertible
$\eqv \exists \{\pi_j\} : \sum_{j=0}^\infty \pi_j < \infty$ such that
\[\pi(B)x_t = \sum_{j=0}^\infty X_{t-j} = w_t\]

Properties
\begin{itemize}
  \item $\ARMA$ causal $\eqv$
    roots of $\phi(z)$ lie outside the unit circle
    \[\psi(z) = \sum_{j=0}^\infty\psi_j z^j = \frac{\theta(z)}{\phi(z)}
    \quad |z| \le 1\]
  \item $\ARMA$ invertible $\eqv$
    roots of $\theta(z)$ lie outside the unit circle
    \[\pi(z) = \sum_{j=0}^\infty\pi_j z^j = \frac{\phi(z)}{\theta(z)}
    \quad |z| \le 1\]
\end{itemize}

Behavior of the ACF and PACF for causal and invertible ARMA models

\begin{center}
  \begin{tabular}{|c|ccc|}
    \hline
    & $\AR$ & $\MA$ & $\ARMA$\\
    \hline
    ACF & tails off & cuts off after lag $q$ & tails off \\
    PACF & cuts off after lag $p$ & tails off $q$ & tails off \\
    \hline
  \end{tabular}
\end{center}

\subsection{Spectral Analysis}

Periodic process
\begin{align*}
  x_t
  &= A \cos(2\pi\omega t + \phi) \\
  &= U_1 \cos(2\pi \omega t) + U_2 \sin(2\pi \omega t)
\end{align*}

\begin{itemize}
  \item Frequency index $\omega$ (cycles per unit time),
    period $1/\omega$
  \item Amplitude $A$
  \item Phase $\phi$
  \item $U_1 = A\cos\phi$ and $U_2 = A\sin\phi$ often normally distributed \rv's
\end{itemize}

Periodic mixture
\[x_t = \sum_{k=1}^q\left( U_{k1}\cos(2\pi\omega_k t)
+ U_{k2}\sin(2\pi\omega_k t)
\right)\]
\begin{itemize}
  \item $U_{k1}, U_{k2}$, for $k=1,\ldots,q$,
    are independent zero-mean \rv's with variances $\sigma_k^2$
  \item $\gamma(h) = \sum_{k=1}^q \sigma_k^2 \cos(2\pi\omega_k h)$
  \item $\gamma(0) = \E{x_t^2} = \sum_{k=1}^q \sigma_k^2$
\end{itemize}

Spectral representation of a periodic process
\begin{align*}
\gamma(h)
&= \sigma^2 \cos(2\pi\omega_0 h) \\
&= \frac{\sigma^2}{2} e^{-2\pi i \omega_0 h}
+ \frac{\sigma^2}{2} e^{2\pi i \omega_0 h}\\
&= \int_{-1/2}^{1/2} e^{2\pi i \omega h} \d{F(\omega)}
\end{align*}

Spectral distribution function
\[F(\omega)= \begin{cases}
  0 & \omega < -\omega_0 \\
  \sigma^2/2 & -\omega \le \omega < \omega_0 \\
  \sigma^2 & \omega \ge \omega_0
\end{cases}\]
\begin{itemize}
  \item $F(-\infty) = F(-1/2) = 0$
  \item $F(\infty) = F(1/2) = \gamma(0)$
\end{itemize}

Spectral density
\[f(\omega) = \sum_{h=-\infty}^\infty \gamma(h) e^{-2\pi i \omega h}
\quad -\frac{1}{2} \le \omega \le \frac{1}{2}\]
\begin{itemize}
  \item Needs $\sum_{h=-\infty}^\infty |\gamma(h)| < \infty
    \imp \gamma(h) = \int_{-1/2}^{1/2} e^{2\pi i \omega h}f(\omega) \d\omega
    \quad h=0,\pm1,\ldots$
  \item $f(\omega) \ge 0$
  \item $f(\omega) = f(-\omega)$
  \item $f(\omega) = f(1-\omega)$
  \item $\gamma(0) = \V{x_t} = \int_{-1/2}^{1/2}f(\omega)\d\omega$
  \item White noise: $f_w(\omega) = \sigma_w^2$
  \item $\ARMA, \phi(B)x_t = \theta(B)w_t$:
    \[f_x(\omega) = \sigma_w^2 \frac{|\theta(e^{-2\pi i
    \omega})|^2}{|\phi(e^{-2\pi i \omega})|^2}\]
    where $\phi(z) = 1 - \sum_{k=1}^p \phi_k z^k$ and
    $\theta(z) = 1 + \sum_{k=1}^q \theta_k z^k$
\end{itemize}

Discrete Fourier Transform (DFT)
\[d(\omega_j) = n^{-1/2} \sum_{i=1}^n x_t e^{-2\pi i\omega_j t}\]

Fourier/Fundamental frequencies
\[\omega_j = j/n\]

Inverse DFT
\[x_t = n^{-1/2} \sum_{j=0}^{n-1} d(\omega_j) e^{2\pi i\omega_j t}\]

Periodogram
\[I(j/n) = |d(j/n)|^2\]

Scaled Periodogram
\begin{align*}
P(j/n)
&= \frac{4}{n}I(j/n) \\
&=\left( \frac{2}{n} \sum_{t=1}^n x_t \cos(2\pi t j/n) \right)^2
+ \left( \frac{2}{n} \sum_{t=1}^n x_t \sin(2\pi t j/n) \right)^2
\end{align*}

\section{Math}

%\subsection{Orthogonal Functions}
%
%$L_2$ space
%\[L_2(a,b) = \left\{ f: [a,b] \to \R, \int_a^b f(x)^2\dx < \infty \right\}\]
%
%Inner Product
%\[\int f(x)g(x)\dx\]
%
%Norm
%\[\|f\| = \sqrt{\int f^2(x) \dx}\]
%
%Orthogonality (for a series of functions $\phi_i$)
%\begin{align*}
%  \int \phi_j^2(x)\dx &= 1 \; \forall j \\
%  \int \phi_i(x)\phi_j(x)\dx &= 0 \; \forall i \neq j
%\end{align*}
%
%An orthogonal sequence $\phi_1, \phi_2,\dots$ is \emph{complete} if the only
%function that is is orthogonal to each $\phi_j$ is the zero function. Then,
%$\phi_1, \phi_2,\dots$ form an \emph{orthogonal basis} in $L_2$:
%\[f \in L_2 \imp f(x) = \sum_{j=1}^\infty \beta_j \phi_j(x)
%\quad \text{where } \beta_j = \int_a^b f(x)\phi_j(x) \dx\]
%
%Cosine Basis
%\begin{align*}
%  \phi_0(x) &= 1 \\
%  \phi_j(x) &= \sqrt{2}\cos(j\pi x) \quad \forall j\ge1
%\end{align*}
%
%\raggedright
%\textsc{Parseval}'s Relation
%\[\|f\|^2 \equiv \int f^2(x)\dx = \sum_{j=1}^\infty \equiv \|\beta\|^2\]
%
%\textsc{Legendre} Polynomials
%\begin{align*}
%  x &\in [-1,1] \\
%  P_0(x) &= 1\\
%  P_1(x) &= x \\
%  P_{j+1}(x) &= \frac{(2j+1)x(P_j(x) - jP_{j-1}(x)}{j+1} \\
%  \phi_j(x) &= \sqrt{(2j+1)/2} P_j(x) \quad \text{orthogonal basis for }
%  L_2(-1,1)
%\end{align*}

\subsection{Gamma Function}
\label{sec:math:gamma}

\begin{itemize}
  \item Ordinary:
    $\displaystyle\Gamma(s) = \int_0^\infty t^{s-1} e^{-t}dt$
  \item Upper incomplete:
    $\displaystyle\Gamma(s,x) = \int_x^\infty t^{s-1} e^{-t}dt$
  \item Lower incomplete:
    $\displaystyle\gamma(s,x) = \int_0^x t^{s-1} e^{-t}dt$
  \item $\Gamma(\alpha + 1) = \alpha \Gamma(\alpha) \qquad \alpha>1$
  \item $\Gamma(n) = (n-1)! \qquad n \in \mathbb N$
  \item $\Gamma(0) = \Gamma(-1) = \infty$
  \item $\Gamma(1/2) = \sqrt{\pi}$
  \item $\Gamma(-1/2) = -2 \Gamma(1/2)$
\end{itemize}

\subsection{Beta Function}
\label{sec:math:beta}

\begin{itemize}
  \item Ordinary: $\text{B}(x,y) = \text{B}(y,x)
    = \displaystyle\int_0^1 t^{x-1}(1-t)^{y-1} \,dt
    = \displaystyle\frac{\Gamma(x)\Gamma(y)}{\Gamma(x+y)}$
%    \item $\alpha,\beta \in \mathbb N \imp \displaystyle
%      \text{B}(\alpha,\beta) = \frac{(\alpha-1)!(\beta-1)!}{(\alpha+\beta-1)!}$
\item Incomplete: $\text{B}(x;\,a,b)
    = \displaystyle\int_0^x t^{a-1}(1-t)^{b-1} \,dt$
  \item Regularized incomplete: \\
    $I_x(a,b) = \displaystyle\frac{\text{B}(x;\,a,b)}{\text{B}(a,b)}
      \stackrel{a,b\in\mathbb N}{=}
      \sum_{j=a}^{a+b-1} \frac{(a+b-1)!}{j!(a+b-1-j)!}x^j(1-x)^{a+b-1-j}$
  \item $I_0(a,b) = 0 \qquad I_1(a,b) = 1$
  \item $I_x(a,b) = 1 - I_{1-x}(b,a)$
\end{itemize}

\subsection{Series}

\begin{multicols}{2}
  \begin{titemize}{Finite}
  \item $\displaystyle\sum_{k=1}^n k = \frac{n(n+1)}{2}$
  \item $\displaystyle\sum_{k=1}^n (2k-1) = n^2$
  \item $\displaystyle\sum_{k=1}^n k^2 = \frac{n(n+1)(2n+1)}{6}$
  \item $\displaystyle\sum_{k=1}^n k^3 = \left(\frac{n(n+1)}{2}\right)^2$
  \item $\displaystyle\sum_{k=0}^n c^k = \frac{c^{n+1}-1}{c-1} \quad c\neq1$
\end{titemize}

\begin{titemize}{Binomial}
  \item $\displaystyle\sum_{k=0}^n \binom{n}{k} = 2^n$
  \item $\displaystyle\sum_{k=0}^n \binom{r+k}{k}=\binom{r+n+1}{n}$
  \item $\displaystyle\sum_{k=0}^n \binom{k}{m}=\binom{n+1}{m+1}$
  \item \textsc{Vandermonde}'s Identity:\\
    $\displaystyle\sum_{k=0}^r \binom{m}{k}\binom{n}{r-k}=\binom{m+n}{r}$
  \item Binomial Theorem:\\
    $\displaystyle\sum_{k=0}^n \binom{n}{k}a^{n-k}b^k = (a+b)^n$
\end{titemize}
\end{multicols}

Infinite
\begin{itemize}
  \item $\displaystyle\sum_{k=0}^\infty p^k = \frac{1}{1-p},
    \quad \sum_{k=1}^\infty p^k = \frac{p}{1-p} \quad |p|<1$
  \item $\displaystyle\sum_{k=0}^\infty kp^{k-1}
    = \displaystyle\frac{d}{dp}\left(\sum_{k=0}^\infty p^k\right)
    = \displaystyle\frac{d}{dp}\left(\frac{1}{1-p}\right)
    = \frac{1}{(1-p)^2} \quad |p|<1$
  \item $\displaystyle\sum_{k=0}^\infty \binom{r+k-1}{k} x^k = (1-x)^{-r}
    \quad r\in\mathbb N^+$
  \item $\displaystyle\sum_{k=0}^\infty \binom{\alpha}{k} p^k
    = (1+p)^\alpha \quad |p|<1\,,\,\alpha \in \mathbb C$
\end{itemize}

%\subsection{Integrals}
%
%\begin{itemize}
%  \item $\displaystyle\int_{-\infty}^\infty e^{-\frac{x^2}{2}}dx
%    = \sqrt{2\pi}$
%\end{itemize}

\vfill~

\subsection{Combinatorics}

Sampling
\begin{center}
  \begin{tabular}[h]{|l*2{|>{\begin{math}\displaystyle}c<{\end{math}}}|}
  \hline &&\\[-1.5ex]
  $k$ out of $n$ & \text{w/o replacement} & \text{w/ replacement}
  \\[1ex]
  \hline
  ordered & n^{\underline k}
    = \displaystyle\prod_{i=0}^{k-1}(n-i)
    = \frac{n!}{(n-k)!}
    & n^k \\[3ex]
    unordered & \binom{n}{k} = \frac{n^{\underline k}}{k!}
    = \frac{n!}{k!(n-k)!} &
    \binom{n-1+r}{r}=\binom{n-1+r}{n-1} \\[3ex]
  \hline
\end{tabular}
\end{center}

\newcommand{\stirling}[2]{\genfrac{\{}{\}}{0pt}{}{#1}{#2}}

Stirling numbers, $2^{nd}$ kind
\[\stirling{n}{k} = k\stirling{n-1}{k}+\stirling{n-1}{k-1}
  \qquad 1\le k \le n \qquad
  \stirling{n}{0} = \begin{cases} 1 & n = 0\\ 0 & \text{else} \end{cases}\]

Partitions
\[P_{n+k,k} = \sum_{i=1}^n P_{n,i} \qquad \qquad
  k>n:\;P_{n,k} = 0 \qquad n\ge1:\;P_{n,0} = 0, \; P_{0,0} = 1\]

% Distinguishability.
\def\distinguishable{\ensuremath{D}\xspace}
\def\indistinguishable{\ensuremath{\neg \distinguishable}\xspace}
Balls and Urns \qquad $f: B \to U$ \qquad
\distinguishable = distinguishable,
\indistinguishable = indistinguishable.
\begin{center}
  \begin{tabular}[h]{|l*4{|>{\begin{math}\displaystyle}c<{\end{math}}}|}
    \hline &&&&\\[-1.5ex]
    $|B|=n$, $|U|=m$ & f \text{ arbitrary} & f \text{ injective} &
    f \text{ surjective} & f \text{ bijective} \\[1ex]
    \hline
    \hline &&&&\\[-2ex]
    $B:\distinguishable,\; U:\distinguishable$ &
      m^n & \begin{cases} m^{\underline n} & m \ge n\\
      0 & \text{else} \end{cases} & m!\,\stirling{n}{m} &
      \begin{cases} n! & m = n\\ 0 & \text{else} \end{cases}\\[3ex]
    \hline &&&&\\[-2ex]
    $B:\indistinguishable,\; U:\distinguishable$ &
      \binom{m+n-1}{n} & \binom{m}{n} &
      \binom{n-1}{m-1} &
      \begin{cases} 1 & m = n\\ 0 & \text{else} \end{cases}\\[3ex]
    \hline &&&&\\[-2ex]
    $B:\distinguishable,\; U:\indistinguishable$ &
      \sum_{k=1}^m \stirling{n}{k} & \begin{cases} 1 &
      m\ge n\\ 0 & \text{else} \end{cases} & \stirling{n}{m} &
      \begin{cases} 1 & m = n\\ 0 & \text{else} \end{cases}\\[3ex]
    \hline &&&&\\[-2ex]
    $B:\indistinguishable,\; U:\indistinguishable$ & \sum_{k=1}^m P_{n,k} &
      \begin{cases} 1 & m \ge n\\ 0 & \text{else} \end{cases} & P_{n,m} &
      \begin{cases} 1 & m = n\\ 0 & \text{else} \end{cases}\\[3ex]
    \hline
  \end{tabular}
\end{center}

%  Convergence
%  \begin{itemize}
%    \item $\displaystyle\sum_{k=1}^\infty a_n$ converges if
%      $\displaystyle\lim_{n \to \infty} \left|\frac{a_{n+1}}{a_n}\right| < 1$
%    \item $\displaystyle\sum_{k=1}^\infty a_n$ diverges if
%      $\displaystyle\lim_{n \to \infty} a_n \neq 0$
%    \item $\displaystyle\sum_{k=1}^\infty n^{-p}$ converges if $p > 1$
%  \end{itemize}

%  \subsection{Calculus}
%
%  Polar Coordinates
%  \begin{itemize}
%    \item $x = r\cos\theta \qquad y = r\sin\theta$
%    \item $r = \sqrt{y^2+x^2}$
%    \item $\theta =
%      \begin{cases}
%        0 & \mbox{if } x = 0 \mbox{ and } y = 0\\
%        \arcsin(\frac{y}{r}) & \mbox{if } x \geq 0 \\
%        -\arcsin(\frac{y}{r}) + \pi & \mbox{if } x < 0\\
%      \end{cases}$
%  \end{itemize}

{
\footnotesize
\bibliographystyle{abbrv}
\bibliography{literature}
\vfill~
}

\end{multicols*}

\newpage

\begin{sidewaysfigure}
  \captionsetup{labelformat=empty,labelsep=none}
  \includegraphics[width=\textwidth]{figs/relationships}
  \caption{Univariate distribution relationships, courtesy Leemis and
  McQueston~\cite{Leemis08}.}
\end{sidewaysfigure}

\end{document}
